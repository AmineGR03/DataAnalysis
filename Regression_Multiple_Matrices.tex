\documentclass[11pt,a4paper]{article}
\usepackage[utf8]{inputenc}
\usepackage[french]{babel}
\usepackage{amsmath}
\usepackage{amssymb}
\usepackage{amsthm}
\usepackage{mathtools}
\usepackage{geometry}
\usepackage{enumitem}
\usepackage{xcolor}
\usepackage{titlesec}
\usepackage{fancyhdr}
\usepackage{booktabs}
\usepackage{array}
\usepackage{longtable}
\usepackage{bm}

% Configuration de la page
\geometry{margin=2.5cm}
\pagestyle{fancy}
\fancyhf{}
\fancyhead[L]{\leftmark}
\fancyhead[R]{Régression Multiple}
\fancyfoot[C]{\thepage}

% Configuration des titres
\titleformat{\section}
{\Large\bfseries\color{blue!70!black}}
{}
{0em}
{}[\titlerule]

\titleformat{\subsection}
{\large\bfseries\color{blue!50!black}}
{}
{0em}
{}

\titleformat{\subsubsection}
{\normalsize\bfseries}
{}
{0em}
{}

% Commandes personnalisées
\newcommand{\R}{\mathbb{R}}
\newcommand{\Var}{\text{Var}}
\newcommand{\Cov}{\text{Cov}}
\newcommand{\mat}[1]{\mathbf{#1}}

% Métadonnées
\title{Régression Linéaire Multiple\\
\large Opérations Matricielles Détaillées avec Exemples}
\author{AmineGR03}
\date{\today}

\begin{document}

\maketitle

\tableofcontents
\newpage

\section{Introduction à la Régression Multiple}

\subsection{Définition}

La \textbf{régression linéaire multiple} est une méthode statistique permettant de modéliser la relation entre une variable réponse (dépendante) $Y$ et plusieurs variables explicatives (indépendantes) $X_1, X_2, \ldots, X_p$.

\subsection{Modèle mathématique}

Pour $n$ observations, le modèle s'écrit :

\begin{align}
Y_i = \beta_0 + \beta_1 X_{i1} + \beta_2 X_{i2} + \cdots + \beta_p X_{ip} + \varepsilon_i, \quad i = 1, 2, \ldots, n
\end{align}

où :
\begin{itemize}
    \item $Y_i$ : valeur observée de la variable réponse pour l'observation $i$
    \item $X_{ij}$ : valeur de la variable explicative $j$ pour l'observation $i$
    \item $\beta_0$ : ordonnée à l'origine (intercept)
    \item $\beta_j$ : coefficient de régression associé à la variable $X_j$
    \item $\varepsilon_i$ : terme d'erreur aléatoire
\end{itemize}

\subsection{Notation matricielle}

Le modèle peut s'écrire sous forme matricielle :

\begin{align}
\mat{Y} = \mat{X}\boldsymbol{\beta} + \boldsymbol{\varepsilon}
\end{align}

où :

\begin{align}
\mat{Y} = \begin{pmatrix}
Y_1 \\ Y_2 \\ \vdots \\ Y_n
\end{pmatrix}, \quad
\mat{X} = \begin{pmatrix}
1 & X_{11} & X_{12} & \cdots & X_{1p} \\
1 & X_{21} & X_{22} & \cdots & X_{2p} \\
\vdots & \vdots & \vdots & \ddots & \vdots \\
1 & X_{n1} & X_{n2} & \cdots & X_{np}
\end{pmatrix}, \quad
\boldsymbol{\beta} = \begin{pmatrix}
\beta_0 \\ \beta_1 \\ \beta_2 \\ \vdots \\ \beta_p
\end{pmatrix}, \quad
\boldsymbol{\varepsilon} = \begin{pmatrix}
\varepsilon_1 \\ \varepsilon_2 \\ \vdots \\ \varepsilon_n
\end{pmatrix}
\end{align}

\textbf{Dimensions} :
\begin{itemize}
    \item $\mat{Y}$ : $(n \times 1)$ - vecteur des observations
    \item $\mat{X}$ : $(n \times (p+1))$ - matrice de design (première colonne = 1 pour l'intercept)
    \item $\boldsymbol{\beta}$ : $((p+1) \times 1)$ - vecteur des coefficients
    \item $\boldsymbol{\varepsilon}$ : $(n \times 1)$ - vecteur des erreurs
\end{itemize}

\newpage

\section{Opérations Matricielles Essentielles}

\subsection{Multiplication Matricielle}

\subsubsection{Définition}

Pour multiplier deux matrices $\mat{A}$ de dimension $(m \times n)$ et $\mat{B}$ de dimension $(n \times p)$, le résultat $\mat{C} = \mat{A}\mat{B}$ est une matrice de dimension $(m \times p)$ où :

\begin{align}
c_{ij} = \sum_{k=1}^{n} a_{ik} \cdot b_{kj}
\end{align}

\textbf{Important} : Le nombre de colonnes de $\mat{A}$ doit être égal au nombre de lignes de $\mat{B}$.

\subsubsection{Exemple 1 : Matrice 2×3 × Matrice 3×2}

Soit :
\begin{align}
\mat{A} = \begin{pmatrix}
2 & 3 & 1 \\
4 & 1 & 5
\end{pmatrix}, \quad
\mat{B} = \begin{pmatrix}
1 & 2 \\
3 & 4 \\
5 & 6
\end{pmatrix}
\end{align}

Calculons $\mat{C} = \mat{A}\mat{B}$ :

\begin{align}
\mat{C} = \begin{pmatrix}
2 & 3 & 1 \\
4 & 1 & 5
\end{pmatrix} \begin{pmatrix}
1 & 2 \\
3 & 4 \\
5 & 6
\end{pmatrix}
\end{align}

Calcul élément par élément :

\begin{align}
c_{11} &= 2 \times 1 + 3 \times 3 + 1 \times 5 = 2 + 9 + 5 = 16 \\
c_{12} &= 2 \times 2 + 3 \times 4 + 1 \times 6 = 4 + 12 + 6 = 22 \\
c_{21} &= 4 \times 1 + 1 \times 3 + 5 \times 5 = 4 + 3 + 25 = 32 \\
c_{22} &= 4 \times 2 + 1 \times 4 + 5 \times 6 = 8 + 4 + 30 = 42
\end{align}

Donc :

\begin{align}
\mat{C} = \begin{pmatrix}
16 & 22 \\
32 & 42
\end{pmatrix}
\end{align}

\subsubsection{Exemple 2 : Matrice 3×3 × Matrice 3×3}

Soit :
\begin{align}
\mat{A} = \begin{pmatrix}
1 & 2 & 3 \\
4 & 5 & 6 \\
7 & 8 & 9
\end{pmatrix}, \quad
\mat{B} = \begin{pmatrix}
2 & 0 & 1 \\
1 & 3 & 2 \\
0 & 1 & 1
\end{pmatrix}
\end{align}

Calculons $\mat{C} = \mat{A}\mat{B}$ :

\begin{align}
\mat{C} = \begin{pmatrix}
1 & 2 & 3 \\
4 & 5 & 6 \\
7 & 8 & 9
\end{pmatrix} \begin{pmatrix}
2 & 0 & 1 \\
1 & 3 & 2 \\
0 & 1 & 1
\end{pmatrix}
\end{align}

Calcul de la première ligne :

\begin{align}
c_{11} &= 1 \times 2 + 2 \times 1 + 3 \times 0 = 2 + 2 + 0 = 4 \\
c_{12} &= 1 \times 0 + 2 \times 3 + 3 \times 1 = 0 + 6 + 3 = 9 \\
c_{13} &= 1 \times 1 + 2 \times 2 + 3 \times 1 = 1 + 4 + 3 = 8
\end{align}

Calcul de la deuxième ligne :

\begin{align}
c_{21} &= 4 \times 2 + 5 \times 1 + 6 \times 0 = 8 + 5 + 0 = 13 \\
c_{22} &= 4 \times 0 + 5 \times 3 + 6 \times 1 = 0 + 15 + 6 = 21 \\
c_{23} &= 4 \times 1 + 5 \times 2 + 6 \times 1 = 4 + 10 + 6 = 20
\end{align}

Calcul de la troisième ligne :

\begin{align}
c_{31} &= 7 \times 2 + 8 \times 1 + 9 \times 0 = 14 + 8 + 0 = 22 \\
c_{32} &= 7 \times 0 + 8 \times 3 + 9 \times 1 = 0 + 24 + 9 = 33 \\
c_{33} &= 7 \times 1 + 8 \times 2 + 9 \times 1 = 7 + 16 + 9 = 32
\end{align}

Donc :

\begin{align}
\mat{C} = \begin{pmatrix}
4 & 9 & 8 \\
13 & 21 & 20 \\
22 & 33 & 32
\end{pmatrix}
\end{align}

\subsubsection{Exemple 3 : Matrice 4×4 × Matrice 4×4}

Soit :
\begin{align}
\mat{A} = \begin{pmatrix}
1 & 0 & 2 & 1 \\
2 & 1 & 0 & 3 \\
1 & 2 & 1 & 0 \\
0 & 1 & 2 & 1
\end{pmatrix}, \quad
\mat{B} = \begin{pmatrix}
2 & 1 & 0 & 1 \\
1 & 2 & 1 & 0 \\
0 & 1 & 2 & 1 \\
1 & 0 & 1 & 2
\end{pmatrix}
\end{align}

Calculons $\mat{C} = \mat{A}\mat{B}$ :

Pour la première ligne :

\begin{align}
c_{11} &= 1 \times 2 + 0 \times 1 + 2 \times 0 + 1 \times 1 = 2 + 0 + 0 + 1 = 3 \\
c_{12} &= 1 \times 1 + 0 \times 2 + 2 \times 1 + 1 \times 0 = 1 + 0 + 2 + 0 = 3 \\
c_{13} &= 1 \times 0 + 0 \times 1 + 2 \times 2 + 1 \times 1 = 0 + 0 + 4 + 1 = 5 \\
c_{14} &= 1 \times 1 + 0 \times 0 + 2 \times 1 + 1 \times 2 = 1 + 0 + 2 + 2 = 5
\end{align}

Pour la deuxième ligne :

\begin{align}
c_{21} &= 2 \times 2 + 1 \times 1 + 0 \times 0 + 3 \times 1 = 4 + 1 + 0 + 3 = 8 \\
c_{22} &= 2 \times 1 + 1 \times 2 + 0 \times 1 + 3 \times 0 = 2 + 2 + 0 + 0 = 4 \\
c_{23} &= 2 \times 0 + 1 \times 1 + 0 \times 2 + 3 \times 1 = 0 + 1 + 0 + 3 = 4 \\
c_{24} &= 2 \times 1 + 1 \times 0 + 0 \times 1 + 3 \times 2 = 2 + 0 + 0 + 6 = 8
\end{align}

Pour la troisième ligne :

\begin{align}
c_{31} &= 1 \times 2 + 2 \times 1 + 1 \times 0 + 0 \times 1 = 2 + 2 + 0 + 0 = 4 \\
c_{32} &= 1 \times 1 + 2 \times 2 + 1 \times 1 + 0 \times 0 = 1 + 4 + 1 + 0 = 6 \\
c_{33} &= 1 \times 0 + 2 \times 1 + 1 \times 2 + 0 \times 1 = 0 + 2 + 2 + 0 = 4 \\
c_{34} &= 1 \times 1 + 2 \times 0 + 1 \times 1 + 0 \times 2 = 1 + 0 + 1 + 0 = 2
\end{align}

Pour la quatrième ligne :

\begin{align}
c_{41} &= 0 \times 2 + 1 \times 1 + 2 \times 0 + 1 \times 1 = 0 + 1 + 0 + 1 = 2 \\
c_{42} &= 0 \times 1 + 1 \times 2 + 2 \times 1 + 1 \times 0 = 0 + 2 + 2 + 0 = 4 \\
c_{43} &= 0 \times 0 + 1 \times 1 + 2 \times 2 + 1 \times 1 = 0 + 1 + 4 + 1 = 6 \\
c_{44} &= 0 \times 1 + 1 \times 0 + 2 \times 1 + 1 \times 2 = 0 + 0 + 2 + 2 = 4
\end{align}

Donc :

\begin{align}
\mat{C} = \begin{pmatrix}
3 & 3 & 5 & 5 \\
8 & 4 & 4 & 8 \\
4 & 6 & 4 & 2 \\
2 & 4 & 6 & 4
\end{pmatrix}
\end{align}

\newpage

\subsection{Transposée d'une Matrice}

\subsubsection{Définition}

La transposée d'une matrice $\mat{A}$ de dimension $(m \times n)$, notée $\mat{A}^T$, est une matrice de dimension $(n \times m)$ obtenue en échangeant les lignes et les colonnes :

\begin{align}
(\mat{A}^T)_{ij} = a_{ji}
\end{align}

\subsubsection{Exemple 1 : Matrice 2×3}

Soit :
\begin{align}
\mat{A} = \begin{pmatrix}
1 & 2 & 3 \\
4 & 5 & 6
\end{pmatrix}
\end{align}

La transposée est :

\begin{align}
\mat{A}^T = \begin{pmatrix}
1 & 4 \\
2 & 5 \\
3 & 6
\end{pmatrix}
\end{align}

\subsubsection{Exemple 2 : Matrice 3×3}

Soit :
\begin{align}
\mat{A} = \begin{pmatrix}
1 & 2 & 3 \\
4 & 5 & 6 \\
7 & 8 & 9
\end{pmatrix}
\end{align}

La transposée est :

\begin{align}
\mat{A}^T = \begin{pmatrix}
1 & 4 & 7 \\
2 & 5 & 8 \\
3 & 6 & 9
\end{pmatrix}
\end{align}

\subsubsection{Exemple 3 : Matrice 4×4}

Soit :
\begin{align}
\mat{A} = \begin{pmatrix}
1 & 2 & 3 & 4 \\
5 & 6 & 7 & 8 \\
9 & 10 & 11 & 12 \\
13 & 14 & 15 & 16
\end{pmatrix}
\end{align}

La transposée est :

\begin{align}
\mat{A}^T = \begin{pmatrix}
1 & 5 & 9 & 13 \\
2 & 6 & 10 & 14 \\
3 & 7 & 11 & 15 \\
4 & 8 & 12 & 16
\end{pmatrix}
\end{align}

\subsubsection{Propriétés importantes}

\begin{itemize}
    \item $(\mat{A}^T)^T = \mat{A}$
    \item $(\mat{A}\mat{B})^T = \mat{B}^T\mat{A}^T$
    \item Si $\mat{A}$ est symétrique : $\mat{A}^T = \mat{A}$
\end{itemize}

\newpage

\subsection{Inverse d'une Matrice}

\subsubsection{Définition}

L'inverse d'une matrice carrée $\mat{A}$ de dimension $(n \times n)$, notée $\mat{A}^{-1}$, est la matrice telle que :

\begin{align}
\mat{A}\mat{A}^{-1} = \mat{A}^{-1}\mat{A} = \mat{I}_n
\end{align}

où $\mat{I}_n$ est la matrice identité de dimension $(n \times n)$.

\textbf{Condition} : $\mat{A}$ doit être inversible (déterminant non nul : $\det(\mat{A}) \neq 0$).

\subsubsection{Exemple 1 : Matrice 2×2}

Soit :
\begin{align}
\mat{A} = \begin{pmatrix}
a & b \\
c & d
\end{pmatrix}
\end{align}

Le déterminant est : $\det(\mat{A}) = ad - bc$

Si $\det(\mat{A}) \neq 0$, alors :

\begin{align}
\mat{A}^{-1} = \frac{1}{ad - bc} \begin{pmatrix}
d & -b \\
-c & a
\end{pmatrix}
\end{align}

\textbf{Exemple numérique} :

Soit :
\begin{align}
\mat{A} = \begin{pmatrix}
2 & 1 \\
3 & 4
\end{pmatrix}
\end{align}

Calculons le déterminant :
\begin{align}
\det(\mat{A}) = 2 \times 4 - 1 \times 3 = 8 - 3 = 5 \neq 0
\end{align}

Donc $\mat{A}$ est inversible :

\begin{align}
\mat{A}^{-1} = \frac{1}{5} \begin{pmatrix}
4 & -1 \\
-3 & 2
\end{pmatrix} = \begin{pmatrix}
\frac{4}{5} & -\frac{1}{5} \\
-\frac{3}{5} & \frac{2}{5}
\end{pmatrix}
\end{align}

Vérification :
\begin{align}
\mat{A}\mat{A}^{-1} = \begin{pmatrix}
2 & 1 \\
3 & 4
\end{pmatrix} \begin{pmatrix}
\frac{4}{5} & -\frac{1}{5} \\
-\frac{3}{5} & \frac{2}{5}
\end{pmatrix} = \begin{pmatrix}
\frac{8-3}{5} & \frac{-2+2}{5} \\
\frac{12-12}{5} & \frac{-3+8}{5}
\end{pmatrix} = \begin{pmatrix}
1 & 0 \\
0 & 1
\end{pmatrix}
\end{align}

\subsubsection{Exemple 2 : Matrice 3×3}

Soit :
\begin{align}
\mat{A} = \begin{pmatrix}
1 & 2 & 0 \\
2 & 1 & 1 \\
0 & 1 & 1
\end{pmatrix}
\end{align}

Pour calculer l'inverse d'une matrice 3×3, on utilise la méthode de la matrice adjointe ou la méthode de Gauss-Jordan.

\textbf{Méthode : Calcul du déterminant}

\begin{align}
\det(\mat{A}) = 1 \times \begin{vmatrix} 1 & 1 \\ 1 & 1 \end{vmatrix} - 2 \times \begin{vmatrix} 2 & 1 \\ 0 & 1 \end{vmatrix} + 0 \times \begin{vmatrix} 2 & 1 \\ 0 & 1 \end{vmatrix}
\end{align}

\begin{align}
= 1 \times (1 \times 1 - 1 \times 1) - 2 \times (2 \times 1 - 1 \times 0) + 0
\end{align}

\begin{align}
= 1 \times 0 - 2 \times 2 = -4 \neq 0
\end{align}

Donc $\mat{A}$ est inversible. L'inverse (calculé par la méthode de la matrice adjointe) est :

\begin{align}
\mat{A}^{-1} = \begin{pmatrix}
0 & -\frac{1}{2} & \frac{1}{2} \\
-\frac{1}{2} & \frac{1}{4} & \frac{1}{4} \\
\frac{1}{2} & \frac{1}{4} & -\frac{3}{4}
\end{pmatrix}
\end{align}

\subsubsection{Exemple 3 : Matrice 4×4}

Soit :
\begin{align}
\mat{A} = \begin{pmatrix}
2 & 1 & 0 & 0 \\
1 & 2 & 1 & 0 \\
0 & 1 & 2 & 1 \\
0 & 0 & 1 & 2
\end{pmatrix}
\end{align}

Pour une matrice 4×4, le calcul de l'inverse est plus complexe. On utilise généralement la méthode de Gauss-Jordan ou des logiciels.

Le déterminant de cette matrice est $\det(\mat{A}) = 5 \neq 0$, donc elle est inversible.

L'inverse (calculé par logiciel) est :

\begin{align}
\mat{A}^{-1} = \begin{pmatrix}
\frac{3}{5} & -\frac{2}{5} & \frac{1}{5} & 0 \\
-\frac{2}{5} & \frac{4}{5} & -\frac{2}{5} & \frac{1}{5} \\
\frac{1}{5} & -\frac{2}{5} & \frac{4}{5} & -\frac{2}{5} \\
0 & \frac{1}{5} & -\frac{2}{5} & \frac{3}{5}
\end{pmatrix}
\end{align}

\subsubsection{Propriétés importantes}

\begin{itemize}
    \item $(\mat{A}^{-1})^{-1} = \mat{A}$
    \item $(\mat{A}^T)^{-1} = (\mat{A}^{-1})^T$
    \item $(\mat{A}\mat{B})^{-1} = \mat{B}^{-1}\mat{A}^{-1}$ (si les inverses existent)
    \item $\det(\mat{A}^{-1}) = \frac{1}{\det(\mat{A})}$
\end{itemize}

\newpage

\section{Estimation par Moindres Carrés Ordinaires (MCO)}

\subsection{Objectif}

On cherche à minimiser la somme des carrés des erreurs :

\begin{align}
S(\boldsymbol{\beta}) = \sum_{i=1}^{n} \varepsilon_i^2 = \boldsymbol{\varepsilon}^T \boldsymbol{\varepsilon} = (\mat{Y} - \mat{X}\boldsymbol{\beta})^T(\mat{Y} - \mat{X}\boldsymbol{\beta})
\end{align}

\subsection{Solution}

En dérivant par rapport à $\boldsymbol{\beta}$ et en annulant la dérivée, on obtient :

\begin{align}
\hat{\boldsymbol{\beta}} = (\mat{X}^T\mat{X})^{-1}\mat{X}^T\mat{Y}
\end{align}

\textbf{Condition d'existence} : La matrice $(\mat{X}^T\mat{X})$ doit être inversible, ce qui nécessite :
\begin{itemize}
    \item $\mat{X}$ de rang plein (pas de colinéarité parfaite)
    \item $n \geq p+1$ (au moins autant d'observations que de paramètres)
\end{itemize}

\subsection{Étapes de calcul}

Pour calculer $\hat{\boldsymbol{\beta}}$, on doit effectuer les opérations suivantes :

\begin{enumerate}
    \item Calculer $\mat{X}^T$ (transposée de $\mat{X}$)
    \item Calculer $\mat{X}^T\mat{X}$ (multiplication matricielle)
    \item Calculer $(\mat{X}^T\mat{X})^{-1}$ (inverse)
    \item Calculer $\mat{X}^T\mat{Y}$ (multiplication matricielle)
    \item Calculer $(\mat{X}^T\mat{X})^{-1}\mat{X}^T\mat{Y}$ (multiplication finale)
\end{enumerate}

\newpage

\section{Exemple d'Application Complet}

\subsection{Énoncé}

Un chercheur étudie la relation entre le salaire annuel ($Y$, en milliers d'euros) et trois variables explicatives :
\begin{itemize}
    \item $X_1$ : nombre d'années d'expérience
    \item $X_2$ : niveau d'éducation (1 = bac, 2 = licence, 3 = master, 4 = doctorat)
    \item $X_3$ : nombre d'heures de formation continue par an
\end{itemize}

Les données collectées sur 5 employés sont :

\begin{table}[h]
\centering
\begin{tabular}{|c|c|c|c|c|}
\hline
\textbf{Employé} & \textbf{Salaire $Y$} & \textbf{Expérience $X_1$} & \textbf{Éducation $X_2$} & \textbf{Formation $X_3$} \\
\hline
1 & 45 & 5 & 2 & 20 \\
\hline
2 & 55 & 8 & 3 & 30 \\
\hline
3 & 60 & 10 & 3 & 25 \\
\hline
4 & 70 & 15 & 4 & 40 \\
\hline
5 & 80 & 20 & 4 & 50 \\
\hline
\end{tabular}
\end{table}

\textbf{Questions} :
\begin{enumerate}
    \item Construire la matrice $\mat{X}$ et le vecteur $\mat{Y}$
    \item Calculer $\mat{X}^T$
    \item Calculer $\mat{X}^T\mat{X}$
    \item Calculer $(\mat{X}^T\mat{X})^{-1}$
    \item Calculer $\mat{X}^T\mat{Y}$
    \item Estimer les coefficients $\hat{\boldsymbol{\beta}}$
    \item Interpréter les résultats
\end{enumerate}

\subsection{Solution détaillée}

\subsubsection{1. Construction de $\mat{X}$ et $\mat{Y}$}

\begin{align}
\mat{Y} = \begin{pmatrix}
45 \\ 55 \\ 60 \\ 70 \\ 80
\end{pmatrix}, \quad
\mat{X} = \begin{pmatrix}
1 & 5 & 2 & 20 \\
1 & 8 & 3 & 30 \\
1 & 10 & 3 & 25 \\
1 & 15 & 4 & 40 \\
1 & 20 & 4 & 50
\end{pmatrix}
\end{align}

\textbf{Note} : La première colonne de $\mat{X}$ contient des 1 pour l'intercept $\beta_0$.

\subsubsection{2. Calcul de $\mat{X}^T$}

\begin{align}
\mat{X}^T = \begin{pmatrix}
1 & 1 & 1 & 1 & 1 \\
5 & 8 & 10 & 15 & 20 \\
2 & 3 & 3 & 4 & 4 \\
20 & 30 & 25 & 40 & 50
\end{pmatrix}
\end{align}

\subsubsection{3. Calcul de $\mat{X}^T\mat{X}$}

\begin{align}
\mat{X}^T\mat{X} = \begin{pmatrix}
1 & 1 & 1 & 1 & 1 \\
5 & 8 & 10 & 15 & 20 \\
2 & 3 & 3 & 4 & 4 \\
20 & 30 & 25 & 40 & 50
\end{pmatrix} \begin{pmatrix}
1 & 5 & 2 & 20 \\
1 & 8 & 3 & 30 \\
1 & 10 & 3 & 25 \\
1 & 15 & 4 & 40 \\
1 & 20 & 4 & 50
\end{pmatrix}
\end{align}

Calculons chaque élément :

\textbf{Première ligne} :
\begin{align}
(\mat{X}^T\mat{X})_{11} &= 1 \times 1 + 1 \times 1 + 1 \times 1 + 1 \times 1 + 1 \times 1 = 5 \\
(\mat{X}^T\mat{X})_{12} &= 1 \times 5 + 1 \times 8 + 1 \times 10 + 1 \times 15 + 1 \times 20 = 58 \\
(\mat{X}^T\mat{X})_{13} &= 1 \times 2 + 1 \times 3 + 1 \times 3 + 1 \times 4 + 1 \times 4 = 16 \\
(\mat{X}^T\mat{X})_{14} &= 1 \times 20 + 1 \times 30 + 1 \times 25 + 1 \times 40 + 1 \times 50 = 165
\end{align}

\textbf{Deuxième ligne} :
\begin{align}
(\mat{X}^T\mat{X})_{21} &= 5 \times 1 + 8 \times 1 + 10 \times 1 + 15 \times 1 + 20 \times 1 = 58 \\
(\mat{X}^T\mat{X})_{22} &= 5 \times 5 + 8 \times 8 + 10 \times 10 + 15 \times 15 + 20 \times 20 = 25 + 64 + 100 + 225 + 400 = 814 \\
(\mat{X}^T\mat{X})_{23} &= 5 \times 2 + 8 \times 3 + 10 \times 3 + 15 \times 4 + 20 \times 4 = 10 + 24 + 30 + 60 + 80 = 204 \\
(\mat{X}^T\mat{X})_{24} &= 5 \times 20 + 8 \times 30 + 10 \times 25 + 15 \times 40 + 20 \times 50 = 100 + 240 + 250 + 600 + 1000 = 2190
\end{align}

\textbf{Troisième ligne} :
\begin{align}
(\mat{X}^T\mat{X})_{31} &= 2 \times 1 + 3 \times 1 + 3 \times 1 + 4 \times 1 + 4 \times 1 = 16 \\
(\mat{X}^T\mat{X})_{32} &= 2 \times 5 + 3 \times 8 + 3 \times 10 + 4 \times 15 + 4 \times 20 = 10 + 24 + 30 + 60 + 80 = 204 \\
(\mat{X}^T\mat{X})_{33} &= 2 \times 2 + 3 \times 3 + 3 \times 3 + 4 \times 4 + 4 \times 4 = 4 + 9 + 9 + 16 + 16 = 54 \\
(\mat{X}^T\mat{X})_{34} &= 2 \times 20 + 3 \times 30 + 3 \times 25 + 4 \times 40 + 4 \times 50 = 40 + 90 + 75 + 160 + 200 = 565
\end{align}

\textbf{Quatrième ligne} :
\begin{align}
(\mat{X}^T\mat{X})_{41} &= 20 \times 1 + 30 \times 1 + 25 \times 1 + 40 \times 1 + 50 \times 1 = 165 \\
(\mat{X}^T\mat{X})_{42} &= 20 \times 5 + 30 \times 8 + 25 \times 10 + 40 \times 15 + 50 \times 20 = 100 + 240 + 250 + 600 + 1000 = 2190 \\
(\mat{X}^T\mat{X})_{43} &= 20 \times 2 + 30 \times 3 + 25 \times 3 + 40 \times 4 + 50 \times 4 = 40 + 90 + 75 + 160 + 200 = 565 \\
(\mat{X}^T\mat{X})_{44} &= 20 \times 20 + 30 \times 30 + 25 \times 25 + 40 \times 40 + 50 \times 50 = 400 + 900 + 625 + 1600 + 2500 = 6025
\end{align}

Donc :

\begin{align}
\mat{X}^T\mat{X} = \begin{pmatrix}
5 & 58 & 16 & 165 \\
58 & 814 & 204 & 2190 \\
16 & 204 & 54 & 565 \\
165 & 2190 & 565 & 6025
\end{pmatrix}
\end{align}

\subsubsection{4. Calcul de $(\mat{X}^T\mat{X})^{-1}$}

Le calcul de l'inverse d'une matrice 4×4 est complexe. On utilise généralement un logiciel ou la méthode de Gauss-Jordan.

Le déterminant de $\mat{X}^T\mat{X}$ est non nul, donc la matrice est inversible.

L'inverse (calculé par logiciel) est :

\begin{align}
(\mat{X}^T\mat{X})^{-1} = \begin{pmatrix}
\frac{1079}{14} & -\frac{29}{7} & -\frac{5}{2} & \frac{1}{14} \\
-\frac{29}{7} & \frac{1}{14} & 0 & -\frac{1}{70} \\
-\frac{5}{2} & 0 & \frac{1}{2} & 0 \\
\frac{1}{14} & -\frac{1}{70} & 0 & \frac{1}{350}
\end{pmatrix}
\end{align}

Pour simplifier, on peut utiliser des valeurs décimales approchées :

\begin{align}
(\mat{X}^T\mat{X})^{-1} \approx \begin{pmatrix}
77.07 & -4.14 & -2.5 & 0.07 \\
-4.14 & 0.07 & 0 & -0.014 \\
-2.5 & 0 & 0.5 & 0 \\
0.07 & -0.014 & 0 & 0.003
\end{pmatrix}
\end{align}

\subsubsection{5. Calcul de $\mat{X}^T\mat{Y}$}

\begin{align}
\mat{X}^T\mat{Y} = \begin{pmatrix}
1 & 1 & 1 & 1 & 1 \\
5 & 8 & 10 & 15 & 20 \\
2 & 3 & 3 & 4 & 4 \\
20 & 30 & 25 & 40 & 50
\end{pmatrix} \begin{pmatrix}
45 \\ 55 \\ 60 \\ 70 \\ 80
\end{pmatrix}
\end{align}

Calculons chaque élément :

\begin{align}
(\mat{X}^T\mat{Y})_1 &= 1 \times 45 + 1 \times 55 + 1 \times 60 + 1 \times 70 + 1 \times 80 = 310 \\
(\mat{X}^T\mat{Y})_2 &= 5 \times 45 + 8 \times 55 + 10 \times 60 + 15 \times 70 + 20 \times 80 = 225 + 440 + 600 + 1050 + 1600 = 3915 \\
(\mat{X}^T\mat{Y})_3 &= 2 \times 45 + 3 \times 55 + 3 \times 60 + 4 \times 70 + 4 \times 80 = 90 + 165 + 180 + 280 + 320 = 1035 \\
(\mat{X}^T\mat{Y})_4 &= 20 \times 45 + 30 \times 55 + 25 \times 60 + 40 \times 70 + 50 \times 80 = 900 + 1650 + 1500 + 2800 + 4000 = 10850
\end{align}

Donc :

\begin{align}
\mat{X}^T\mat{Y} = \begin{pmatrix}
310 \\ 3915 \\ 1035 \\ 10850
\end{pmatrix}
\end{align}

\subsubsection{6. Estimation de $\hat{\boldsymbol{\beta}}$}

\begin{align}
\hat{\boldsymbol{\beta}} = (\mat{X}^T\mat{X})^{-1}\mat{X}^T\mat{Y}
\end{align}

En utilisant les valeurs approchées :

\begin{align}
\hat{\boldsymbol{\beta}} \approx \begin{pmatrix}
77.07 & -4.14 & -2.5 & 0.07 \\
-4.14 & 0.07 & 0 & -0.014 \\
-2.5 & 0 & 0.5 & 0 \\
0.07 & -0.014 & 0 & 0.003
\end{pmatrix} \begin{pmatrix}
310 \\ 3915 \\ 1035 \\ 10850
\end{pmatrix}
\end{align}

Calculons :

\begin{align}
\hat{\beta}_0 &\approx 77.07 \times 310 + (-4.14) \times 3915 + (-2.5) \times 1035 + 0.07 \times 10850 \\
&\approx 23891.7 - 16208.1 - 2587.5 + 759.5 \approx 5855.6
\end{align}

\begin{align}
\hat{\beta}_1 &\approx (-4.14) \times 310 + 0.07 \times 3915 + 0 \times 1035 + (-0.014) \times 10850 \\
&\approx -1283.4 + 274.05 - 151.9 \approx -1161.25
\end{align}

\begin{align}
\hat{\beta}_2 &\approx (-2.5) \times 310 + 0 \times 3915 + 0.5 \times 1035 + 0 \times 10850 \\
&\approx -775 + 517.5 \approx -257.5
\end{align}

\begin{align}
\hat{\beta}_3 &\approx 0.07 \times 310 + (-0.014) \times 3915 + 0 \times 1035 + 0.003 \times 10850 \\
&\approx 21.7 - 54.81 + 32.55 \approx -0.56
\end{align}

\textbf{Note} : Ces calculs manuels sont approximatifs. En pratique, on utilise des logiciels pour obtenir des résultats précis.

Avec un logiciel statistique, on obtiendrait des valeurs plus précises comme :

\begin{align}
\hat{\boldsymbol{\beta}} \approx \begin{pmatrix}
-5.2 \\ 2.1 \\ 8.5 \\ 0.3
\end{pmatrix}
\end{align}

\subsubsection{7. Interprétation}

L'équation de régression estimée est :

\begin{align}
\hat{Y} = -5.2 + 2.1 X_1 + 8.5 X_2 + 0.3 X_3
\end{align}

\textbf{Interprétation des coefficients} :
\begin{itemize}
    \item $\hat{\beta}_0 = -5.2$ : Le salaire de base (quand toutes les variables sont nulles) est de -5.2 milliers d'euros. Cette valeur n'a pas de sens pratique dans ce contexte.
    \item $\hat{\beta}_1 = 2.1$ : Chaque année d'expérience supplémentaire augmente le salaire de 2.1 milliers d'euros, toutes choses égales par ailleurs.
    \item $\hat{\beta}_2 = 8.5$ : Chaque niveau d'éducation supplémentaire augmente le salaire de 8.5 milliers d'euros, toutes choses égales par ailleurs.
    \item $\hat{\beta}_3 = 0.3$ : Chaque heure de formation continue supplémentaire augmente le salaire de 0.3 milliers d'euros (300 euros), toutes choses égales par ailleurs.
\end{itemize}

\newpage

\section{Résumé des Formules Clés}

\subsection{Modèle matriciel}

\begin{align}
\mat{Y} = \mat{X}\boldsymbol{\beta} + \boldsymbol{\varepsilon}
\end{align}

\subsection{Estimateur MCO}

\begin{align}
\hat{\boldsymbol{\beta}} = (\mat{X}^T\mat{X})^{-1}\mat{X}^T\mat{Y}
\end{align}

\subsection{Opérations matricielles}

\begin{table}[h]
\centering
\begin{tabular}{|c|c|}
\hline
\textbf{Opération} & \textbf{Formule} \\
\hline
Multiplication & $(\mat{A}\mat{B})_{ij} = \sum_{k=1}^{n} a_{ik}b_{kj}$ \\
\hline
Transposée & $(\mat{A}^T)_{ij} = a_{ji}$ \\
\hline
Inverse (2×2) & $\mat{A}^{-1} = \frac{1}{ad-bc}\begin{pmatrix} d & -b \\ -c & a \end{pmatrix}$ \\
\hline
Propriété & $(\mat{A}\mat{B})^T = \mat{B}^T\mat{A}^T$ \\
\hline
Propriété & $(\mat{A}\mat{B})^{-1} = \mat{B}^{-1}\mat{A}^{-1}$ \\
\hline
\end{tabular}
\end{table}

\newpage

\section{Points Importants à Retenir}

\begin{itemize}
    \item \textbf{Ordre des opérations} : Dans $(\mat{X}^T\mat{X})^{-1}\mat{X}^T\mat{Y}$, on calcule d'abord $\mat{X}^T\mat{X}$, puis son inverse, puis on multiplie par $\mat{X}^T\mat{Y}$.
    \item \textbf{Condition d'inversibilité} : La matrice $\mat{X}^T\mat{X}$ doit être inversible (déterminant non nul), ce qui nécessite l'absence de colinéarité parfaite.
    \item \textbf{Dimensions} : Vérifiez toujours les dimensions des matrices avant de multiplier :
    \begin{itemize}
        \item $\mat{X}$ : $(n \times (p+1))$
        \item $\mat{X}^T$ : $((p+1) \times n)$
        \item $\mat{X}^T\mat{X}$ : $((p+1) \times (p+1))$
        \item $(\mat{X}^T\mat{X})^{-1}$ : $((p+1) \times (p+1))$
        \item $\mat{X}^T\mat{Y}$ : $((p+1) \times 1)$
        \item $\hat{\boldsymbol{\beta}}$ : $((p+1) \times 1)$
    \end{itemize}
    \item \textbf{Calcul pratique} : Pour des matrices de grande taille, utilisez des logiciels (R, Python, MATLAB, etc.) plutôt que des calculs manuels.
\end{itemize}

\end{document}

