\documentclass[11pt,a4paper]{article}
\usepackage[utf8]{inputenc}
\usepackage[french]{babel}
\usepackage{amsmath}
\usepackage{amssymb}
\usepackage{geometry}
\usepackage{enumitem}
\usepackage{xcolor}
\usepackage{titlesec}
\usepackage{fancyhdr}
\usepackage{booktabs}
\usepackage{array}
\usepackage{listings}

% Configuration de la page
\geometry{margin=2.5cm}
\pagestyle{fancy}
\fancyhf{}
\fancyhead[L]{\leftmark}
\fancyhead[R]{Redis}
\fancyfoot[C]{\thepage}

% Configuration des titres
\titleformat{\section}
{\Large\bfseries\color{red!70!black}}
{}
{0em}
{}[\titlerule]

\titleformat{\subsection}
{\large\bfseries\color{red!50!black}}
{}
{0em}
{}

\titleformat{\subsubsection}
{\normalsize\bfseries}
{}
{0em}
{}

% Configuration pour le code
\lstset{
    language=bash,
    basicstyle=\ttfamily\small,
    keywordstyle=\color{blue}\bfseries,
    commentstyle=\color{gray},
    stringstyle=\color{red},
    numbers=left,
    numberstyle=\tiny\color{gray},
    stepnumber=1,
    numbersep=5pt,
    backgroundcolor=\color{gray!10},
    frame=single,
    breaklines=true,
    showstringspaces=false,
    tabsize=2
}

% Métadonnées
\title{Résumé Complet Redis\\
\large CRUD, Structures de Données, Commandes}
\author{AmineGR03}
\date{\today}

\begin{document}

\maketitle

\tableofcontents
\newpage

\section{Opérations CRUD de Base}

\subsection{String (Chaîne)}

\subsubsection{Create (SET)}

\begin{lstlisting}
SET cle "valeur"
SET nom "Dupont"
SET age 30
\end{lstlisting}

\subsubsection{Read (GET)}

\begin{lstlisting}
GET cle
GET nom
GET age
\end{lstlisting}

\subsubsection{Update (SET)}

\begin{lstlisting}
SET nom "Martin"  -- Remplace la valeur
\end{lstlisting}

\subsubsection{Delete (DEL)}

\begin{lstlisting}
DEL cle
DEL nom age  -- Supprime plusieurs clés
\end{lstlisting}

\subsubsection{Opérations Avancées sur String}

\begin{lstlisting}
-- SET avec expiration (secondes)
SET cle "valeur" EX 3600

-- SET avec expiration (millisecondes)
SET cle "valeur" PX 3600000

-- SET si n'existe pas (NX)
SET cle "valeur" NX

-- SET si existe (XX)
SET cle "valeur" XX

-- GETSET (récupère et remplace)
GETSET cle "nouvelle_valeur"

-- APPEND
APPEND cle "_suffixe"

-- INCR (incrémenter)
INCR compteur

-- INCRBY (incrémenter de n)
INCRBY compteur 5

-- DECR (décrémenter)
DECR compteur

-- DECRBY (décrémenter de n)
DECRBY compteur 3

-- MGET (multiple GET)
MGET cle1 cle2 cle3

-- MSET (multiple SET)
MSET cle1 "valeur1" cle2 "valeur2" cle3 "valeur3"
\end{lstlisting}

\subsection{Hash (Tableau Associatif)}

\subsubsection{Create (HSET)}

\begin{lstlisting}
HSET utilisateur:1 nom "Dupont"
HSET utilisateur:1 age 30
HSET utilisateur:1 ville "Paris"
\end{lstlisting}

\subsubsection{Read (HGET, HGETALL)}

\begin{lstlisting}
HGET utilisateur:1 nom
HGETALL utilisateur:1
\end{lstlisting}

\subsubsection{Update (HSET)}

\begin{lstlisting}
HSET utilisateur:1 age 31
\end{lstlisting}

\subsubsection{Delete (HDEL)}

\begin{lstlisting}
HDEL utilisateur:1 ville
HDEL utilisateur:1 nom age  -- Supprime plusieurs champs
\end{lstlisting}

\subsubsection{Opérations Avancées sur Hash}

\begin{lstlisting}
-- HMSET (multiple SET)
HMSET utilisateur:1 nom "Dupont" age 30 ville "Paris"

-- HMGET (multiple GET)
HMGET utilisateur:1 nom age ville

-- HKEYS (toutes les clés)
HKEYS utilisateur:1

-- HVALS (toutes les valeurs)
HVALS utilisateur:1

-- HEXISTS (vérifier existence)
HEXISTS utilisateur:1 nom

-- HLEN (nombre de champs)
HLEN utilisateur:1

-- HINCRBY (incrémenter champ numérique)
HINCRBY utilisateur:1 age 1

-- HINCRBYFLOAT
HINCRBYFLOAT utilisateur:1 score 0.5
\end{lstlisting}

\subsection{List (Liste)}

\subsubsection{Create (LPUSH, RPUSH)}

\begin{lstlisting}
LPUSH liste "element1"
LPUSH liste "element2" "element3"
RPUSH liste "element4"
\end{lstlisting}

\subsubsection{Read (LRANGE, LINDEX)}

\begin{lstlisting}
LRANGE liste 0 -1  -- Tous les éléments
LRANGE liste 0 2   -- 3 premiers éléments
LINDEX liste 0     -- Premier élément
LINDEX liste -1    -- Dernier élément
\end{lstlisting}

\subsubsection{Update (LSET)}

\begin{lstlisting}
LSET liste 0 "nouvelle_valeur"
\end{lstlisting}

\subsubsection{Delete (LPOP, RPOP, LREM)}

\begin{lstlisting}
LPOP liste          -- Retire le premier
RPOP liste          -- Retire le dernier
LREM liste 2 "val"  -- Retire 2 occurrences de "val"
\end{lstlisting}

\subsubsection{Opérations Avancées sur List}

\begin{lstlisting}
-- LLEN (longueur)
LLEN liste

-- LTRIM (garder seulement un range)
LTRIM liste 0 2

-- RPOPLPUSH (déplacer de fin à début d'une autre liste)
RPOPLPUSH liste1 liste2

-- BLPOP (blocking left pop)
BLPOP liste 10  -- Attend jusqu'à 10 secondes

-- BRPOP (blocking right pop)
BRPOP liste 10
\end{lstlisting}

\subsection{Set (Ensemble)}

\subsubsection{Create (SADD)}

\begin{lstlisting}
SADD tags "important"
SADD tags "urgent" "prioritaire"
\end{lstlisting}

\subsubsection{Read (SMEMBERS, SISMEMBER)}

\begin{lstlisting}
SMEMBERS tags
SISMEMBER tags "important"  -- Retourne 1 si existe, 0 sinon
\end{lstlisting}

\subsubsection{Update (SADD pour ajouter)}

\begin{lstlisting}
SADD tags "nouveau_tag"
\end{lstlisting}

\subsubsection{Delete (SREM)}

\begin{lstlisting}
SREM tags "important"
SREM tags "urgent" "prioritaire"
\end{lstlisting}

\subsubsection{Opérations Avancées sur Set}

\begin{lstlisting}
-- SCARD (cardinalité)
SCARD tags

-- SRANDMEMBER (élément aléatoire)
SRANDMEMBER tags
SRANDMEMBER tags 3  -- 3 éléments aléatoires

-- SPOP (retirer un élément aléatoire)
SPOP tags

-- SDIFF (différence)
SDIFF set1 set2

-- SINTER (intersection)
SINTER set1 set2

-- SUNION (union)
SUNION set1 set2

-- SDIFFSTORE (différence et stocker)
SDIFFSTORE resultat set1 set2

-- SINTERSTORE (intersection et stocker)
SINTERSTORE resultat set1 set2

-- SUNIONSTORE (union et stocker)
SUNIONSTORE resultat set1 set2

-- SMOVE (déplacer élément)
SMOVE set1 set2 "element"
\end{lstlisting}

\subsection{Sorted Set (Ensemble Ordonné)}

\subsubsection{Create (ZADD)}

\begin{lstlisting}
ZADD scores 100 "joueur1"
ZADD scores 200 "joueur2" 150 "joueur3"
\end{lstlisting}

\subsubsection{Read (ZRANGE, ZREVRANGE)}

\begin{lstlisting}
ZRANGE scores 0 -1                    -- Tous, par score croissant
ZRANGE scores 0 -1 WITHSCORES        -- Avec scores
ZREVRANGE scores 0 -1                -- Par score décroissant
ZREVRANGE scores 0 2                 -- Top 3
ZSCORE scores "joueur1"               -- Score d'un membre
\end{lstlisting}

\subsubsection{Update (ZADD pour modifier score)}

\begin{lstlisting}
ZADD scores 250 "joueur1"  -- Met à jour le score
ZINCRBY scores 10 "joueur1"  -- Incrémente le score
\end{lstlisting}

\subsubsection{Delete (ZREM)}

\begin{lstlisting}
ZREM scores "joueur1"
ZREM scores "joueur2" "joueur3"
\end{lstlisting}

\subsubsection{Opérations Avancées sur Sorted Set}

\begin{lstlisting}
-- ZCARD (nombre d'éléments)
ZCARD scores

-- ZCOUNT (compter dans un range de scores)
ZCOUNT scores 100 200

-- ZRANK (rang d'un membre, croissant)
ZRANK scores "joueur1"

-- ZREVRANK (rang d'un membre, décroissant)
ZREVRANK scores "joueur1"

-- ZRANGEBYSCORE (par range de scores)
ZRANGEBYSCORE scores 100 200
ZRANGEBYSCORE scores 100 200 WITHSCORES LIMIT 0 10

-- ZREVRANGEBYSCORE
ZREVRANGEBYSCORE scores 200 100

-- ZREMRANGEBYRANK (supprimer par rang)
ZREMRANGEBYRANK scores 0 2  -- Supprime les 3 premiers

-- ZREMRANGEBYSCORE (supprimer par score)
ZREMRANGEBYSCORE scores 0 100

-- ZUNIONSTORE (union de sorted sets)
ZUNIONSTORE resultat 2 set1 set2 WEIGHTS 1 2

-- ZINTERSTORE (intersection de sorted sets)
ZINTERSTORE resultat 2 set1 set2
\end{lstlisting}

\newpage

\section{Gestion des Clés}

\subsection{Opérations sur Clés}

\begin{lstlisting}
-- EXISTS (vérifier existence)
EXISTS cle

-- TYPE (type de la clé)
TYPE cle

-- TTL (time to live en secondes)
TTL cle

-- PTTL (time to live en millisecondes)
PTTL cle

-- EXPIRE (définir expiration en secondes)
EXPIRE cle 3600

-- PEXPIRE (définir expiration en millisecondes)
PEXPIRE cle 3600000

-- EXPIREAT (expiration à un timestamp)
EXPIREAT cle 1234567890

-- PERSIST (supprimer expiration)
PERSIST cle

-- RENAME (renommer)
RENAME ancienne nouvelle

-- RENAMENX (renommer si nouvelle n'existe pas)
RENAMENX ancienne nouvelle

-- KEYS (trouver toutes les clés correspondant au pattern)
KEYS utilisateur:*

-- SCAN (scanner les clés, recommandé)
SCAN 0 MATCH utilisateur:* COUNT 100

-- DEL (supprimer)
DEL cle1 cle2 cle3

-- UNLINK (supprimer de manière asynchrone)
UNLINK cle1 cle2
\end{lstlisting}

\subsection{Patterns de Clés}

\begin{lstlisting}
-- * : correspond à n'importe quel caractère
KEYS utilisateur:*

-- ? : correspond à un seul caractère
KEYS utilisateur:?

-- [abc] : correspond à a, b ou c
KEYS utilisateur:[123]

-- [a-z] : correspond à un caractère dans la plage
KEYS utilisateur:[1-9]
\end{lstlisting}

\newpage

\section{Transactions}

\subsection{MULTI/EXEC}

\begin{lstlisting}
MULTI
SET cle1 "valeur1"
SET cle2 "valeur2"
INCR compteur
EXEC
\end{lstlisting}

\subsection{DISCARD}

\begin{lstlisting}
MULTI
SET cle1 "valeur1"
DISCARD  -- Annule la transaction
\end{lstlisting}

\subsection{WATCH}

\begin{lstlisting}
WATCH cle
MULTI
SET cle "valeur"
EXEC  -- Échoue si cle a été modifiée entre WATCH et EXEC
\end{lstlisting}

\newpage

\section{Pub/Sub (Publication/Souscription)}

\subsection{PUBLISH}

\begin{lstlisting}
PUBLISH canal "message"
\end{lstlisting}

\subsection{SUBSCRIBE}

\begin{lstlisting}
SUBSCRIBE canal1 canal2
\end{lstlisting}

\subsection{PSUBSCRIBE (Pattern Subscribe)}

\begin{lstlisting}
PSUBSCRIBE news.*
\end{lstlisting}

\subsection{UNSUBSCRIBE}

\begin{lstlisting}
UNSUBSCRIBE canal1
\end{lstlisting}

\subsection{PUNSUBSCRIBE}

\begin{lstlisting}
PUNSUBSCRIBE news.*
\end{lstlisting}

\newpage

\section{Exemples Pratiques}

\subsection{Exemple 1 : Système de Votes}

\begin{lstlisting}
-- Voter pour un article
ZINCRBY article:votes 1 "article:123"

-- Obtenir les top articles
ZREVRANGE article:votes 0 9 WITHSCORES

-- Vérifier le score d'un article
ZSCORE article:votes "article:123"
\end{lstlisting}

\subsection{Exemple 2 : Cache Utilisateur}

\begin{lstlisting}
-- Stocker un utilisateur
HMSET utilisateur:123 nom "Dupont" age 30 ville "Paris"
EXPIRE utilisateur:123 3600

-- Récupérer un utilisateur
HGETALL utilisateur:123

-- Mettre à jour
HSET utilisateur:123 age 31
\end{lstlisting}

\subsection{Exemple 3 : File d'Attente}

\begin{lstlisting}
-- Ajouter à la file
LPUSH file:travaux "travail1"
LPUSH file:travaux "travail2"

-- Traiter un travail
RPOP file:travaux

-- Voir la file
LRANGE file:travaux 0 -1
\end{lstlisting}

\subsection{Exemple 4 : Tags d'Articles}

\begin{lstlisting}
-- Ajouter des tags à un article
SADD article:123:tags "tech" "programming" "redis"

-- Trouver tous les articles avec un tag
SMEMBERS article:123:tags

-- Trouver les articles communs (intersection)
SINTER article:123:tags article:456:tags
\end{lstlisting}

\subsection{Exemple 5 : Leaderboard}

\begin{lstlisting}
-- Ajouter un score
ZADD leaderboard 1000 "joueur1"
ZADD leaderboard 2000 "joueur2"
ZADD leaderboard 1500 "joueur3"

-- Top 10
ZREVRANGE leaderboard 0 9 WITHSCORES

-- Rang d'un joueur
ZREVRANK leaderboard "joueur1"

-- Incrémenter le score
ZINCRBY leaderboard 100 "joueur1"
\end{lstlisting}

\newpage

\section{Résumé des Opérations Essentielles}

\subsection{Structures de Données}

\begin{table}[h]
\centering
\begin{tabular}{|c|c|c|}
\hline
\textbf{Type} & \textbf{Create} & \textbf{Read} \\
\hline
String & \texttt{SET} & \texttt{GET} \\
\hline
Hash & \texttt{HSET} & \texttt{HGET}, \texttt{HGETALL} \\
\hline
List & \texttt{LPUSH}, \texttt{RPUSH} & \texttt{LRANGE}, \texttt{LINDEX} \\
\hline
Set & \texttt{SADD} & \texttt{SMEMBERS} \\
\hline
Sorted Set & \texttt{ZADD} & \texttt{ZRANGE}, \texttt{ZREVRANGE} \\
\hline
\end{tabular}
\end{table}

\subsection{Opérations CRUD par Type}

\begin{table}[h]
\centering
\begin{tabular}{|c|c|c|c|c|}
\hline
\textbf{Type} & \textbf{Create} & \textbf{Read} & \textbf{Update} & \textbf{Delete} \\
\hline
String & \texttt{SET} & \texttt{GET} & \texttt{SET} & \texttt{DEL} \\
\hline
Hash & \texttt{HSET} & \texttt{HGET} & \texttt{HSET} & \texttt{HDEL} \\
\hline
List & \texttt{LPUSH/RPUSH} & \texttt{LRANGE} & \texttt{LSET} & \texttt{LPOP/RPOP} \\
\hline
Set & \texttt{SADD} & \texttt{SMEMBERS} & \texttt{SADD} & \texttt{SREM} \\
\hline
Sorted Set & \texttt{ZADD} & \texttt{ZRANGE} & \texttt{ZADD} & \texttt{ZREM} \\
\hline
\end{tabular}
\end{table}

\subsection{Commandes Utiles}

\begin{table}[h]
\centering
\begin{tabular}{|c|c|}
\hline
\textbf{Commande} & \textbf{Description} \\
\hline
\texttt{EXISTS} & Vérifier existence d'une clé \\
\hline
\texttt{TYPE} & Type d'une clé \\
\hline
\texttt{TTL} & Time to live \\
\hline
\texttt{EXPIRE} & Définir expiration \\
\hline
\texttt{KEYS} & Trouver des clés (pattern) \\
\hline
\texttt{SCAN} & Scanner les clés (recommandé) \\
\hline
\texttt{DEL} & Supprimer une clé \\
\hline
\texttt{MULTI/EXEC} & Transaction \\
\hline
\texttt{WATCH} & Surveiller une clé \\
\hline
\end{tabular}
\end{table}

\subsection{Points Importants}

\begin{itemize}
    \item \textbf{String} : Valeur simple, peut être n'importe quel type
    \item \textbf{Hash} : Structure clé-valeur imbriquée, idéal pour objets
    \item \textbf{List} : Liste ordonnée, peut être utilisée comme file
    \item \textbf{Set} : Ensemble non ordonné, opérations d'ensemble
    \item \textbf{Sorted Set} : Ensemble ordonné par score, idéal pour leaderboards
    \item \textbf{TTL} : Toutes les structures peuvent avoir une expiration
    \item \textbf{Transactions} : MULTI/EXEC pour exécuter plusieurs commandes atomiquement
    \item \textbf{Pub/Sub} : Système de messagerie asynchrone
    \item \textbf{KEYS vs SCAN} : SCAN est recommandé pour la production (non-bloquant)
\end{itemize}

\end{document}







