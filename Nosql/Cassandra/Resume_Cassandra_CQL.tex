\documentclass[11pt,a4paper]{article}
\usepackage[utf8]{inputenc}
\usepackage[french]{babel}
\usepackage{amsmath}
\usepackage{amssymb}
\usepackage{geometry}
\usepackage{enumitem}
\usepackage{xcolor}
\usepackage{titlesec}
\usepackage{fancyhdr}
\usepackage{booktabs}
\usepackage{array}
\usepackage{listings}

% Configuration de la page
\geometry{margin=2.5cm}
\pagestyle{fancy}
\fancyhf{}
\fancyhead[L]{\leftmark}
\fancyhead[R]{Cassandra CQL}
\fancyfoot[C]{\thepage}

% Configuration des titres
\titleformat{\section}
{\Large\bfseries\color{blue!70!black}}
{}
{0em}
{}[\titlerule]

\titleformat{\subsection}
{\large\bfseries\color{blue!50!black}}
{}
{0em}
{}

\titleformat{\subsubsection}
{\normalsize\bfseries}
{}
{0em}
{}

% Configuration pour le code
\lstset{
    language=SQL,
    basicstyle=\ttfamily\small,
    keywordstyle=\color{blue}\bfseries,
    commentstyle=\color{gray},
    stringstyle=\color{red},
    numbers=left,
    numberstyle=\tiny\color{gray},
    stepnumber=1,
    numbersep=5pt,
    backgroundcolor=\color{gray!10},
    frame=single,
    breaklines=true,
    showstringspaces=false,
    tabsize=2,
    morekeywords={CREATE, TABLE, KEYSPACE, INSERT, SELECT, UPDATE, DELETE, ALTER, DROP, PRIMARY, KEY, WHERE, AND, OR, IN, SET, IF, EXISTS, NOT, NULL, WITH, USING, TTL, TIMESTAMP, BATCH, BEGIN, APPLY, TRUNCATE, INDEX, ON, ALLOW, FILTERING, LIMIT, ORDER, BY, ASC, DESC, COUNT, DISTINCT, GROUP, AS, TOKEN, ALLOW, FILTERING}
}

% Métadonnées
\title{Résumé Complet Cassandra CQL\\
\large CRUD, Requêtes, Filtrage}
\author{AmineGR03}
\date{\today}

\begin{document}

\maketitle

\tableofcontents
\newpage

\section{Création de Structure}

\subsection{Keyspace}

\begin{lstlisting}
CREATE KEYSPACE ma_base
WITH REPLICATION = {
    'class': 'SimpleStrategy',
    'replication_factor': 3
};

USE ma_base;
\end{lstlisting}

\subsection{Table}

\begin{lstlisting}
CREATE TABLE utilisateurs (
    id UUID PRIMARY KEY,
    nom TEXT,
    email TEXT,
    age INT,
    ville TEXT,
    date_creation TIMESTAMP
);
\end{lstlisting}

\subsection{Table avec Clustering Key}

\begin{lstlisting}
CREATE TABLE commandes (
    client_id UUID,
    date_commande TIMESTAMP,
    produit TEXT,
    quantite INT,
    prix DECIMAL,
    PRIMARY KEY (client_id, date_commande, produit)
);
\end{lstlisting}

\subsection{Table avec Partition Key Composite}

\begin{lstlisting}
CREATE TABLE articles (
    categorie TEXT,
    sous_categorie TEXT,
    id UUID,
    titre TEXT,
    prix DECIMAL,
    PRIMARY KEY ((categorie, sous_categorie), id)
);
\end{lstlisting}

\newpage

\section{Opérations CRUD}

\subsection{Create (INSERT)}

\subsubsection{Insertion simple}

\begin{lstlisting}
INSERT INTO utilisateurs (id, nom, email, age, ville)
VALUES (uuid(), 'Dupont', 'dupont@email.com', 30, 'Paris');
\end{lstlisting}

\subsubsection{Insertion avec UUID spécifique}

\begin{lstlisting}
INSERT INTO utilisateurs (id, nom, email, age, ville)
VALUES (
    a1b2c3d4-e5f6-7890-abcd-ef1234567890,
    'Martin',
    'martin@email.com',
    25,
    'Lyon'
);
\end{lstlisting}

\subsubsection{Insertion avec TTL (Time To Live)}

\begin{lstlisting}
INSERT INTO utilisateurs (id, nom, email)
VALUES (uuid(), 'Bernard', 'bernard@email.com')
USING TTL 3600;  -- Expire après 1 heure
\end{lstlisting}

\subsubsection{Insertion avec TIMESTAMP}

\begin{lstlisting}
INSERT INTO utilisateurs (id, nom, date_creation)
VALUES (uuid(), 'Durand', toTimestamp(now()));
\end{lstlisting}

\subsubsection{Insertion conditionnelle (IF NOT EXISTS)}

\begin{lstlisting}
INSERT INTO utilisateurs (id, nom, email)
VALUES (a1b2c3d4-e5f6-7890-abcd-ef1234567890, 'Dupont', 'dupont@email.com')
IF NOT EXISTS;
\end{lstlisting}

\subsection{Read (SELECT)}

\subsubsection{Sélection de tous les champs}

\begin{lstlisting}
SELECT * FROM utilisateurs;
\end{lstlisting}

\subsubsection{Sélection de champs spécifiques}

\begin{lstlisting}
SELECT id, nom, email FROM utilisateurs;
\end{lstlisting}

\subsubsection{Sélection avec WHERE}

\begin{lstlisting}
SELECT * FROM utilisateurs WHERE id = a1b2c3d4-e5f6-7890-abcd-ef1234567890;
\end{lstlisting}

\subsubsection{Sélection avec LIMIT}

\begin{lstlisting}
SELECT * FROM utilisateurs LIMIT 10;
\end{lstlisting}

\subsubsection{Sélection avec COUNT}

\begin{lstlisting}
SELECT COUNT(*) FROM utilisateurs;
\end{lstlisting}

\subsubsection{Sélection avec DISTINCT}

\begin{lstlisting}
SELECT DISTINCT ville FROM utilisateurs;
\end{lstlisting}

\subsection{Update (UPDATE)}

\subsubsection{Mise à jour simple}

\begin{lstlisting}
UPDATE utilisateurs
SET age = 31, ville = 'Lyon'
WHERE id = a1b2c3d4-e5f6-7890-abcd-ef1234567890;
\end{lstlisting}

\subsubsection{Mise à jour avec TTL}

\begin{lstlisting}
UPDATE utilisateurs
USING TTL 7200
SET email = 'nouveau@email.com'
WHERE id = a1b2c3d4-e5f6-7890-abcd-ef1234567890;
\end{lstlisting}

\subsubsection{Mise à jour conditionnelle (IF)}

\begin{lstlisting}
UPDATE utilisateurs
SET age = 32
WHERE id = a1b2c3d4-e5f6-7890-abcd-ef1234567890
IF age = 31;
\end{lstlisting}

\subsubsection{Mise à jour avec IF EXISTS}

\begin{lstlisting}
UPDATE utilisateurs
SET age = 33
WHERE id = a1b2c3d4-e5f6-7890-abcd-ef1234567890
IF EXISTS;
\end{lstlisting}

\subsubsection{Opérations sur Collections}

\begin{lstlisting}
-- Ajouter à une liste
UPDATE utilisateurs
SET hobbies = hobbies + ['lecture']
WHERE id = a1b2c3d4-e5f6-7890-abcd-ef1234567890;

-- Retirer d'une liste
UPDATE utilisateurs
SET hobbies = hobbies - ['sport']
WHERE id = a1b2c3d4-e5f6-7890-abcd-ef1234567890;

-- Ajouter à un set
UPDATE utilisateurs
SET tags = tags + {'important'}
WHERE id = a1b2c3d4-e5f6-7890-abcd-ef1234567890;

-- Mettre à jour une map
UPDATE utilisateurs
SET adresse = {'rue': '123 Main St', 'ville': 'Paris'}
WHERE id = a1b2c3d4-e5f6-7890-abcd-ef1234567890;
\end{lstlisting}

\subsection{Delete (DELETE)}

\subsubsection{Suppression d'un document}

\begin{lstlisting}
DELETE FROM utilisateurs
WHERE id = a1b2c3d4-e5f6-7890-abcd-ef1234567890;
\end{lstlisting}

\subsubsection{Suppression de champs spécifiques}

\begin{lstlisting}
DELETE email, age FROM utilisateurs
WHERE id = a1b2c3d4-e5f6-7890-abcd-ef1234567890;
\end{lstlisting}

\subsubsection{Suppression d'éléments dans une collection}

\begin{lstlisting}
DELETE hobbies[0] FROM utilisateurs
WHERE id = a1b2c3d4-e5f6-7890-abcd-ef1234567890;
\end{lstlisting}

\subsubsection{Suppression conditionnelle}

\begin{lstlisting}
DELETE FROM utilisateurs
WHERE id = a1b2c3d4-e5f6-7890-abcd-ef1234567890
IF age = 30;
\end{lstlisting}

\subsubsection{TRUNCATE (vider une table)}

\begin{lstlisting}
TRUNCATE utilisateurs;
\end{lstlisting}

\newpage

\section{Filtrage et Requêtes}

\subsection{Opérateurs de Comparaison}

\begin{lstlisting}
-- Égalité
SELECT * FROM utilisateurs WHERE age = 30;

-- Plus grand que
SELECT * FROM utilisateurs WHERE age > 25;

-- Plus grand ou égal
SELECT * FROM utilisateurs WHERE age >= 25;

-- Plus petit que
SELECT * FROM utilisateurs WHERE age < 40;

-- Plus petit ou égal
SELECT * FROM utilisateurs WHERE age <= 40;

-- Différent de
SELECT * FROM utilisateurs WHERE age != 30;
\end{lstlisting}

\subsection{Opérateur IN}

\begin{lstlisting}
SELECT * FROM utilisateurs
WHERE ville IN ('Paris', 'Lyon', 'Marseille');
\end{lstlisting}

\subsection{Opérateurs Logiques}

\begin{lstlisting}
-- ET
SELECT * FROM utilisateurs
WHERE age > 25 AND ville = 'Paris';

-- OU
SELECT * FROM utilisateurs
WHERE age < 30 OR ville = 'Lyon';
\end{lstlisting}

\subsection{Opérateur LIKE}

\begin{lstlisting}
-- Commence par
SELECT * FROM utilisateurs
WHERE nom LIKE 'Dup%';

-- Contient
SELECT * FROM utilisateurs
WHERE email LIKE '%@gmail.com';
\end{lstlisting}

\subsection{ALLOW FILTERING}

\begin{lstlisting}
-- Permet de filtrer sur des colonnes non-indexées
-- ATTENTION: Peut être très coûteux en performance
SELECT * FROM utilisateurs
WHERE ville = 'Paris' AND age > 25
ALLOW FILTERING;
\end{lstlisting}

\subsection{Requêtes sur Clustering Columns}

\begin{lstlisting}
-- Requête sur partition key
SELECT * FROM commandes
WHERE client_id = a1b2c3d4-e5f6-7890-abcd-ef1234567890;

-- Requête avec clustering key
SELECT * FROM commandes
WHERE client_id = a1b2c3d4-e5f6-7890-abcd-ef1234567890
AND date_commande > '2024-01-01';

-- Requête avec range sur clustering key
SELECT * FROM commandes
WHERE client_id = a1b2c3d4-e5f6-7890-abcd-ef1234567890
AND date_commande >= '2024-01-01'
AND date_commande <= '2024-12-31';
\end{lstlisting}

\newpage

\section{Collections}

\subsection{List}

\begin{lstlisting}
CREATE TABLE utilisateurs (
    id UUID PRIMARY KEY,
    nom TEXT,
    hobbies LIST<TEXT>
);

-- Insertion
INSERT INTO utilisateurs (id, nom, hobbies)
VALUES (uuid(), 'Dupont', ['lecture', 'sport', 'musique']);

-- Mise à jour
UPDATE utilisateurs
SET hobbies = hobbies + ['cinema']
WHERE id = a1b2c3d4-e5f6-7890-abcd-ef1234567890;

-- Retirer un élément
UPDATE utilisateurs
SET hobbies = hobbies - ['sport']
WHERE id = a1b2c3d4-e5f6-7890-abcd-ef1234567890;
\end{lstlisting}

\subsection{Set}

\begin{lstlisting}
CREATE TABLE articles (
    id UUID PRIMARY KEY,
    titre TEXT,
    tags SET<TEXT>
);

-- Insertion
INSERT INTO articles (id, titre, tags)
VALUES (uuid(), 'Article 1', {'tech', 'programming', 'database'});

-- Ajouter un élément
UPDATE articles
SET tags = tags + {'important'}
WHERE id = a1b2c3d4-e5f6-7890-abcd-ef1234567890;
\end{lstlisting}

\subsection{Map}

\begin{lstlisting}
CREATE TABLE utilisateurs (
    id UUID PRIMARY KEY,
    nom TEXT,
    adresse MAP<TEXT, TEXT>
);

-- Insertion
INSERT INTO utilisateurs (id, nom, adresse)
VALUES (uuid(), 'Dupont', {
    'rue': '123 Main St',
    'ville': 'Paris',
    'codePostal': '75001'
});

-- Mise à jour
UPDATE utilisateurs
SET adresse['ville'] = 'Lyon'
WHERE id = a1b2c3d4-e5f6-7890-abcd-ef1234567890;
\end{lstlisting}

\newpage

\section{Index}

\subsection{Index Secondaire}

\begin{lstlisting}
CREATE INDEX idx_ville ON utilisateurs (ville);

-- Utilisation
SELECT * FROM utilisateurs WHERE ville = 'Paris';
\end{lstlisting}

\subsection{Index sur Collection}

\begin{lstlisting}
CREATE INDEX idx_tags ON articles (tags);

-- Recherche dans un set
SELECT * FROM articles WHERE tags CONTAINS 'tech';
\end{lstlisting}

\subsection{Index sur Clé de Map}

\begin{lstlisting}
CREATE INDEX idx_ville_adresse ON utilisateurs (adresse);

-- Recherche
SELECT * FROM utilisateurs WHERE adresse['ville'] = 'Paris';
\end{lstlisting}

\subsection{Index sur Clé de Map (KEYS)}

\begin{lstlisting}
CREATE INDEX idx_adresse_keys ON utilisateurs (KEYS(adresse));
\end{lstlisting}

\subsection{Index sur Valeur de Map (VALUES)}

\begin{lstlisting}
CREATE INDEX idx_adresse_values ON utilisateurs (VALUES(adresse));
\end{lstlisting}

\subsection{Index sur Entrées de Map (ENTRIES)}

\begin{lstlisting}
CREATE INDEX idx_adresse_entries ON utilisateurs (ENTRIES(adresse));
\end{lstlisting}

\newpage

\section{Types de Données}

\subsection{Types Primitifs}

\begin{lstlisting}
-- Text
nom TEXT

-- Integer
age INT

-- Big Integer
grand_nombre BIGINT

-- Decimal
prix DECIMAL

-- Float
valeur FLOAT

-- Double
valeur_double DOUBLE

-- Boolean
est_actif BOOLEAN

-- UUID
id UUID

-- TimeUUID
timestamp TIMEUUID
\end{lstlisting}

\subsection{Types Temporels}

\begin{lstlisting}
-- Date
date_naissance DATE

-- Time
heure TIME

-- Timestamp
date_creation TIMESTAMP
\end{lstlisting}

\subsection{Types Collections}

\begin{lstlisting}
-- Liste
hobbies LIST<TEXT>

-- Set
tags SET<INT>

-- Map
adresse MAP<TEXT, TEXT>
\end{lstlisting}

\subsection{Types Utilisateur (UDT)}

\begin{lstlisting}
-- Création d'un type
CREATE TYPE adresse_type (
    rue TEXT,
    ville TEXT,
    code_postal TEXT
);

-- Utilisation
CREATE TABLE utilisateurs (
    id UUID PRIMARY KEY,
    nom TEXT,
    adresse adresse_type
);
\end{lstlisting}

\newpage

\section{Fonctions}

\subsection{Fonctions de Conversion}

\begin{lstlisting}
-- UUID vers String
SELECT toText(id) FROM utilisateurs;

-- String vers UUID
SELECT toUUID('a1b2c3d4-e5f6-7890-abcd-ef1234567890') FROM utilisateurs;

-- Timestamp
SELECT toTimestamp(now()) FROM utilisateurs;
\end{lstlisting}

\subsection{Fonctions Temporelles}

\begin{lstlisting}
-- Date actuelle
SELECT now() FROM utilisateurs;

-- Timestamp Unix
SELECT toUnixTimestamp(date_creation) FROM utilisateurs;

-- Depuis Unix timestamp
SELECT toTimestamp(1234567890) FROM utilisateurs;
\end{lstlisting}

\subsection{Fonctions sur Collections}

\begin{lstlisting}
-- Taille d'une liste
SELECT size(hobbies) FROM utilisateurs;

-- Taille d'un set
SELECT size(tags) FROM articles;

-- Taille d'une map
SELECT size(adresse) FROM utilisateurs;
\end{lstlisting}

\newpage

\section{BATCH}

\subsection{Batch Simple}

\begin{lstlisting}
BEGIN BATCH
    INSERT INTO utilisateurs (id, nom, email)
    VALUES (uuid(), 'Dupont', 'dupont@email.com');
    
    INSERT INTO utilisateurs (id, nom, email)
    VALUES (uuid(), 'Martin', 'martin@email.com');
    
    UPDATE utilisateurs
    SET age = 30
    WHERE id = a1b2c3d4-e5f6-7890-abcd-ef1234567890;
APPLY BATCH;
\end{lstlisting}

\subsection{Batch avec Timestamp}

\begin{lstlisting}
BEGIN BATCH
    USING TIMESTAMP 1234567890
    INSERT INTO utilisateurs (id, nom)
    VALUES (uuid(), 'Dupont');
APPLY BATCH;
\end{lstlisting}

\subsection{Batch avec TTL}

\begin{lstlisting}
BEGIN BATCH
    USING TTL 3600
    INSERT INTO utilisateurs (id, nom)
    VALUES (uuid(), 'Dupont');
APPLY BATCH;
\end{lstlisting}

\newpage

\section{Requêtes Avancées}

\subsection{ORDER BY}

\begin{lstlisting}
-- Tri sur clustering column
SELECT * FROM commandes
WHERE client_id = a1b2c3d4-e5f6-7890-abcd-ef1234567890
ORDER BY date_commande DESC;
\end{lstlisting}

\subsection{GROUP BY}

\begin{lstlisting}
-- Groupement (nécessite ALLOW FILTERING)
SELECT ville, COUNT(*) FROM utilisateurs
GROUP BY ville
ALLOW FILTERING;
\end{lstlisting}

\subsection{Requêtes avec Token}

\begin{lstlisting}
-- Requête par range de tokens
SELECT * FROM utilisateurs
WHERE token(id) > token(minTimeuuid('2024-01-01'))
AND token(id) <= token(maxTimeuuid('2024-12-31'));
\end{lstlisting}

\subsection{Requêtes sur Collections}

\begin{lstlisting}
-- Contient dans une liste
SELECT * FROM utilisateurs
WHERE hobbies CONTAINS 'lecture';

-- Contient dans un set
SELECT * FROM articles
WHERE tags CONTAINS 'tech';

-- Clé existe dans une map
SELECT * FROM utilisateurs
WHERE adresse CONTAINS KEY 'ville';

-- Valeur existe dans une map
SELECT * FROM utilisateurs
WHERE adresse CONTAINS 'Paris';
\end{lstlisting}

\newpage

\section{Modification de Structure}

\subsection{ALTER TABLE}

\begin{lstlisting}
-- Ajouter une colonne
ALTER TABLE utilisateurs ADD telephone TEXT;

-- Supprimer une colonne
ALTER TABLE utilisateurs DROP telephone;

-- Modifier le type d'une colonne
ALTER TABLE utilisateurs ALTER telephone TYPE TEXT;
\end{lstlisting}

\subsection{DROP}

\begin{lstlisting}
-- Supprimer une table
DROP TABLE utilisateurs;

-- Supprimer un keyspace
DROP KEYSPACE ma_base;

-- Supprimer un index
DROP INDEX idx_ville;
\end{lstlisting}

\newpage

\section{Résumé des Opérations Essentielles}

\subsection{CRUD}

\begin{table}[h]
\centering
\begin{tabular}{|c|c|}
\hline
\textbf{Opération} & \textbf{Commande} \\
\hline
Insert & \texttt{INSERT INTO ... VALUES ...} \\
\hline
Read & \texttt{SELECT ... FROM ... WHERE ...} \\
\hline
Update & \texttt{UPDATE ... SET ... WHERE ...} \\
\hline
Delete & \texttt{DELETE FROM ... WHERE ...} \\
\hline
\end{tabular}
\end{table}

\subsection{Opérateurs de Filtrage}

\begin{table}[h]
\centering
\begin{tabular}{|c|c|}
\hline
\textbf{Opérateur} & \textbf{Description} \\
\hline
\texttt{=} & Égalité \\
\hline
\texttt{>} & Plus grand que \\
\hline
\texttt{>=} & Plus grand ou égal \\
\hline
\texttt{<} & Plus petit que \\
\hline
\texttt{<=} & Plus petit ou égal \\
\hline
\texttt{!=} & Différent de \\
\hline
\texttt{IN} & Dans une liste \\
\hline
\texttt{LIKE} & Pattern matching \\
\hline
\texttt{CONTAINS} & Contient (collections) \\
\hline
\texttt{CONTAINS KEY} & Clé existe (map) \\
\hline
\end{tabular}
\end{table}

\subsection{Collections}

\begin{table}[h]
\centering
\begin{tabular}{|c|c|}
\hline
\textbf{Type} & \textbf{Description} \\
\hline
\texttt{LIST<T>} & Liste ordonnée \\
\hline
\texttt{SET<T>} & Ensemble non ordonné \\
\hline
\texttt{MAP<K, V>} & Dictionnaire clé-valeur \\
\hline
\end{tabular}
\end{table}

\subsection{Points Importants}

\begin{itemize}
    \item \textbf{Partition Key} : Détermine sur quel nœud les données sont stockées
    \item \textbf{Clustering Key} : Détermine l'ordre des données dans une partition
    \item \textbf{ALLOW FILTERING} : Permet de filtrer sur des colonnes non-indexées (coûteux)
    \item \textbf{TTL} : Permet d'expirer automatiquement les données
    \item \textbf{BATCH} : Permet d'exécuter plusieurs opérations atomiquement
    \item \textbf{IF EXISTS / IF NOT EXISTS} : Conditions pour INSERT/UPDATE/DELETE
\end{itemize}

\end{document}







