\documentclass[11pt,a4paper]{article}
\usepackage[utf8]{inputenc}
\usepackage[french]{babel}
\usepackage{amsmath}
\usepackage{amssymb}
\usepackage{geometry}
\usepackage{enumitem}
\usepackage{xcolor}
\usepackage{titlesec}
\usepackage{fancyhdr}
\usepackage{booktabs}
\usepackage{array}
\usepackage{listings}
\usepackage{xcolor}

% Configuration de la page
\geometry{margin=2.5cm}
\pagestyle{fancy}
\fancyhf{}
\fancyhead[L]{\leftmark}
\fancyhead[R]{MongoDB}
\fancyfoot[C]{\thepage}

% Configuration des titres
\titleformat{\section}
{\Large\bfseries\color{green!70!black}}
{}
{0em}
{}[\titlerule]

\titleformat{\subsection}
{\large\bfseries\color{green!50!black}}
{}
{0em}
{}

\titleformat{\subsubsection}
{\normalsize\bfseries}
{}
{0em}
{}

% Configuration pour le code
\lstset{
    language=JavaScript,
    basicstyle=\ttfamily\small,
    keywordstyle=\color{blue}\bfseries,
    commentstyle=\color{gray},
    stringstyle=\color{red},
    numbers=left,
    numberstyle=\tiny\color{gray},
    stepnumber=1,
    numbersep=5pt,
    backgroundcolor=\color{gray!10},
    frame=single,
    breaklines=true,
    showstringspaces=false,
    tabsize=2
}

% Métadonnées
\title{Résumé Complet MongoDB\\
\large CRUD, Filtrage, Pipeline, Agrégation}
\author{AmineGR03}
\date{\today}

\begin{document}

\maketitle

\tableofcontents
\newpage

\section{Opérations CRUD}

\subsection{Create (Insertion)}

\subsubsection{Insertion d'un document}

\begin{lstlisting}
db.collection.insertOne({
    nom: "Dupont",
    age: 30,
    ville: "Paris"
})
\end{lstlisting}

\subsubsection{Insertion de plusieurs documents}

\begin{lstlisting}
db.collection.insertMany([
    {nom: "Martin", age: 25, ville: "Lyon"},
    {nom: "Bernard", age: 35, ville: "Marseille"}
])
\end{lstlisting}

\subsubsection{Insertion avec \texttt{insert()}}

\begin{lstlisting}
db.collection.insert({
    nom: "Durand",
    age: 28,
    ville: "Toulouse"
})
\end{lstlisting}

\subsection{Read (Lecture)}

\subsubsection{Lecture de tous les documents}

\begin{lstlisting}
db.collection.find()
\end{lstlisting}

\subsubsection{Lecture avec formatage}

\begin{lstlisting}
db.collection.find().pretty()
\end{lstlisting}

\subsubsection{Lecture d'un seul document}

\begin{lstlisting}
db.collection.findOne({age: 30})
\end{lstlisting}

\subsubsection{Projection (sélection de champs)}

\begin{lstlisting}
// Inclure seulement nom et age
db.collection.find({}, {nom: 1, age: 1, _id: 0})

// Exclure ville
db.collection.find({}, {ville: 0})
\end{lstlisting}

\subsection{Update (Mise à jour)}

\subsubsection{Mise à jour d'un document}

\begin{lstlisting}
db.collection.updateOne(
    {nom: "Dupont"},
    {$set: {age: 31}}
)
\end{lstlisting}

\subsubsection{Mise à jour de plusieurs documents}

\begin{lstlisting}
db.collection.updateMany(
    {ville: "Paris"},
    {$set: {pays: "France"}}
)
\end{lstlisting}

\subsubsection{Opérateurs de mise à jour}

\begin{lstlisting}
// $set : définir une valeur
{$set: {age: 30}}

// $inc : incrémenter
{$inc: {age: 1}}

// $unset : supprimer un champ
{$unset: {ville: ""}}

// $push : ajouter à un tableau
{$push: {hobbies: "lecture"}}

// $pull : retirer d'un tableau
{$pull: {hobbies: "lecture"}}

// $addToSet : ajouter si n'existe pas
{$addToSet: {tags: "important"}}
\end{lstlisting}

\subsubsection{Replace (remplacement)}

\begin{lstlisting}
db.collection.replaceOne(
    {nom: "Dupont"},
    {nom: "Dupont", age: 32, ville: "Lyon"}
)
\end{lstlisting}

\subsection{Delete (Suppression)}

\subsubsection{Suppression d'un document}

\begin{lstlisting}
db.collection.deleteOne({nom: "Dupont"})
\end{lstlisting}

\subsubsection{Suppression de plusieurs documents}

\begin{lstlisting}
db.collection.deleteMany({ville: "Paris"})
\end{lstlisting}

\subsubsection{Suppression de tous les documents}

\begin{lstlisting}
db.collection.deleteMany({})
\end{lstlisting}

\newpage

\section{Filtrage et Requêtes}

\subsection{Opérateurs de Comparaison}

\begin{lstlisting}
// Égalité
db.collection.find({age: 30})

// Plus grand que
db.collection.find({age: {$gt: 25}})

// Plus grand ou égal
db.collection.find({age: {$gte: 25}})

// Plus petit que
db.collection.find({age: {$lt: 40}})

// Plus petit ou égal
db.collection.find({age: {$lte: 40}})

// Différent de
db.collection.find({age: {$ne: 30}})

// Dans une liste
db.collection.find({ville: {$in: ["Paris", "Lyon"]}})

// Pas dans une liste
db.collection.find({ville: {$nin: ["Paris", "Lyon"]}})
\end{lstlisting}

\subsection{Opérateurs Logiques}

\begin{lstlisting}
// ET (AND)
db.collection.find({
    $and: [
        {age: {$gt: 25}},
        {ville: "Paris"}
    ]
})

// OU (OR)
db.collection.find({
    $or: [
        {age: {$lt: 30}},
        {ville: "Lyon"}
    ]
})

// NON (NOT)
db.collection.find({
    $not: {age: {$gt: 30}}
})

// NOR
db.collection.find({
    $nor: [
        {age: {$lt: 25}},
        {ville: "Paris"}
    ]
})
\end{lstlisting}

\subsection{Opérateurs pour Tableaux}

\begin{lstlisting}
// Tous les éléments du tableau correspondent
db.collection.find({
    tags: {$all: ["important", "urgent"]}
})

// Taille du tableau
db.collection.find({
    hobbies: {$size: 3}
})

// Élément existe dans le tableau
db.collection.find({
    hobbies: "lecture"
})

// Élément à une position spécifique
db.collection.find({
    "hobbies.0": "lecture"
})
\end{lstlisting}

\subsection{Opérateurs pour Chaînes de Caractères}

\begin{lstlisting}
// Commence par
db.collection.find({
    nom: {$regex: "^Dup", $options: "i"}
})

// Contient
db.collection.find({
    nom: {$regex: "pont", $options: "i"}
})

// Se termine par
db.collection.find({
    nom: {$regex: "ont$", $options: "i"}
})
\end{lstlisting}

\subsection{Opérateur Exists}

\begin{lstlisting}
// Champ existe
db.collection.find({
    email: {$exists: true}
})

// Champ n'existe pas
db.collection.find({
    email: {$exists: false}
})
\end{lstlisting}

\subsection{Opérateur Type}

\begin{lstlisting}
// Vérifier le type
db.collection.find({
    age: {$type: "number"}
})

// Types possibles: double, string, object, array, bool, date, null, int, long, decimal
\end{lstlisting}

\newpage

\section{Requêtes Imbriquées}

\subsection{Accès aux Champs Imbriqués}

\begin{lstlisting}
// Document avec structure imbriquée
{
    nom: "Dupont",
    adresse: {
        rue: "123 Main St",
        ville: "Paris",
        codePostal: "75001"
    }
}

// Requête sur champ imbriqué
db.collection.find({
    "adresse.ville": "Paris"
})

// Mise à jour d'un champ imbriqué
db.collection.updateOne(
    {nom: "Dupont"},
    {$set: {"adresse.codePostal": "75002"}}
)
\end{lstlisting}

\subsection{Tableaux d'Objets}

\begin{lstlisting}
// Document avec tableau d'objets
{
    nom: "Dupont",
    commandes: [
        {produit: "Livre", prix: 20},
        {produit: "Stylo", prix: 5}
    ]
}

// Recherche dans tableau d'objets
db.collection.find({
    "commandes.prix": {$gt: 10}
})

// Recherche avec $elemMatch
db.collection.find({
    commandes: {
        $elemMatch: {
            produit: "Livre",
            prix: {$gt: 15}
        }
    }
})
\end{lstlisting}

\subsection{Requêtes Complexes Imbriquées}

\begin{lstlisting}
// Structure complexe
{
    entreprise: {
        nom: "TechCorp",
        employes: [
            {
                nom: "Martin",
                departement: {
                    nom: "IT",
                    budget: 100000
                }
            }
        ]
    }
}

// Requête multi-niveaux
db.collection.find({
    "entreprise.employes.departement.budget": {$gt: 50000}
})

// Avec $elemMatch
db.collection.find({
    "entreprise.employes": {
        $elemMatch: {
            "departement.nom": "IT",
            "departement.budget": {$gt: 50000}
        }
    }
})
\end{lstlisting}

\newpage

\section{Tri et Limitation}

\subsection{Tri}

\begin{lstlisting}
// Tri croissant
db.collection.find().sort({age: 1})

// Tri décroissant
db.collection.find().sort({age: -1})

// Tri multiple
db.collection.find().sort({ville: 1, age: -1})
\end{lstlisting}

\subsection{Limitation}

\begin{lstlisting}
// Limiter le nombre de résultats
db.collection.find().limit(10)

// Ignorer les premiers résultats
db.collection.find().skip(5)

// Combinaison
db.collection.find()
    .sort({age: -1})
    .skip(10)
    .limit(5)
\end{lstlisting}

\newpage

\section{Pipeline d'Agrégation}

\subsection{Introduction}

Le pipeline d'agrégation permet de traiter et transformer des documents à travers plusieurs étapes.

\subsection{Stages Principaux}

\subsubsection{\$match - Filtrage}

\begin{lstlisting}
db.collection.aggregate([
    {
        $match: {
            age: {$gt: 25},
            ville: "Paris"
        }
    }
])
\end{lstlisting}

\subsubsection{\$project - Projection et Transformation}

\begin{lstlisting}
db.collection.aggregate([
    {
        $project: {
            nom: 1,
            age: 1,
            ageAnneeProchaine: {$add: ["$age", 1]},
            estMajeur: {$gte: ["$age", 18]},
            _id: 0
        }
    }
])
\end{lstlisting}

\subsubsection{\$group - Regroupement}

\begin{lstlisting}
db.collection.aggregate([
    {
        $group: {
            _id: "$ville",
            nombrePersonnes: {$sum: 1},
            ageMoyen: {$avg: "$age"},
            ageMin: {$min: "$age"},
            ageMax: {$max: "$age"}
        }
    }
])
\end{lstlisting}

\subsubsection{\$sort - Tri}

\begin{lstlisting}
db.collection.aggregate([
    {
        $sort: {age: -1}
    }
])
\end{lstlisting}

\subsubsection{\$limit - Limitation}

\begin{lstlisting}
db.collection.aggregate([
    {
        $limit: 10
    }
])
\end{lstlisting}

\subsubsection{\$skip - Ignorer}

\begin{lstlisting}
db.collection.aggregate([
    {
        $skip: 5
    }
])
\end{lstlisting}

\subsubsection{\$unwind - Déplier un Tableau}

\begin{lstlisting}
// Avant unwind
{
    nom: "Dupont",
    hobbies: ["lecture", "sport", "musique"]
}

// Après unwind
db.collection.aggregate([
    {
        $unwind: "$hobbies"
    }
])

// Résultat: 3 documents séparés, un pour chaque hobby
\end{lstlisting}

\subsubsection{\$lookup - Jointure}

\begin{lstlisting}
db.commandes.aggregate([
    {
        $lookup: {
            from: "clients",
            localField: "clientId",
            foreignField: "_id",
            as: "clientInfo"
        }
    }
])
\end{lstlisting}

\subsubsection{\$addFields - Ajouter des Champs}

\begin{lstlisting}
db.collection.aggregate([
    {
        $addFields: {
            agePlusUn: {$add: ["$age", 1]},
            nomComplet: {
                $concat: ["$prenom", " ", "$nom"]
            }
        }
    }
])
\end{lstlisting}

\subsubsection{\$count - Compter}

\begin{lstlisting}
db.collection.aggregate([
    {
        $match: {age: {$gt: 25}}
    },
    {
        $count: "nombrePersonnes"
    }
])
\end{lstlisting}

\newpage

\section{Opérateurs d'Agrégation}

\subsection{Opérateurs Arithmétiques}

\begin{lstlisting}
// Addition
{$add: ["$prix", "$taxe"]}

// Soustraction
{$subtract: ["$prix", "$remise"]}

// Multiplication
{$multiply: ["$quantite", "$prixUnitaire"]}

// Division
{$divide: ["$total", "$nombre"]}

// Modulo
{$mod: ["$nombre", 2]}

// Puissance
{$pow: ["$base", "$exposant"]}

// Racine carrée
{$sqrt: "$nombre"}

// Valeur absolue
{$abs: "$difference"}
\end{lstlisting}

\subsection{Opérateurs de Comparaison}

\begin{lstlisting}
// Égalité
{$eq: ["$a", "$b"]}

// Plus grand que
{$gt: ["$a", "$b"]}

// Plus grand ou égal
{$gte: ["$a", "$b"]}

// Plus petit que
{$lt: ["$a", "$b"]}

// Plus petit ou égal
{$lte: ["$a", "$b"]}

// Différent de
{$ne: ["$a", "$b"]}

// Comparaison multiple
{$cmp: ["$a", "$b"]}  // Retourne -1, 0, ou 1
\end{lstlisting}

\subsection{Opérateurs Logiques}

\begin{lstlisting}
// ET
{$and: [condition1, condition2]}

// OU
{$or: [condition1, condition2]}

// NON
{$not: condition}

// Condition ternaire
{$cond: {
    if: condition,
    then: valeurSiVrai,
    else: valeurSiFaux
}}
\end{lstlisting}

\subsection{Opérateurs pour Chaînes}

\begin{lstlisting}
// Concaténation
{$concat: ["$prenom", " ", "$nom"]}

// Sous-chaîne
{$substr: ["$texte", 0, 5]}

// Longueur
{$strLenCP: "$texte"}

// Convertir en minuscules
{$toLower: "$texte"}

// Convertir en majuscules
{$toUpper: "$texte"}
\end{lstlisting}

\subsection{Opérateurs de Date}

\begin{lstlisting}
// Année
{$year: "$date"}

// Mois
{$month: "$date"}

// Jour
{$dayOfMonth: "$date"}

// Jour de la semaine
{$dayOfWeek: "$date"}

// Heure
{$hour: "$date"}

// Différence entre dates (en millisecondes)
{$subtract: ["$dateFin", "$dateDebut"]}
\end{lstlisting}

\subsection{Opérateurs de Tableau}

\begin{lstlisting}
// Taille du tableau
{$size: "$tableau"}

// Filtrer un tableau
{$filter: {
    input: "$tableau",
    as: "item",
    cond: {$gt: ["$$item.prix", 10]}
}}

// Prendre les n premiers éléments
{$slice: ["$tableau", 3]}

// Concaténer des tableaux
{$concatArrays: ["$tableau1", "$tableau2"]}
\end{lstlisting}

\newpage

\section{Agrégation Avancée}

\subsection{Group avec Accumulateurs}

\begin{lstlisting}
db.collection.aggregate([
    {
        $group: {
            _id: "$ville",
            // Compter
            nombre: {$sum: 1},
            
            // Somme
            totalAge: {$sum: "$age"},
            
            // Moyenne
            ageMoyen: {$avg: "$age"},
            
            // Minimum
            ageMin: {$min: "$age"},
            
            // Maximum
            ageMax: {$max: "$age"},
            
            // Premier élément
            premier: {$first: "$nom"},
            
            // Dernier élément
            dernier: {$last: "$nom"},
            
            // Ajouter à un tableau
            tousLesNoms: {$push: "$nom"},
            
            // Ajouter à un ensemble
            nomsUniques: {$addToSet: "$nom"},
            
            // Écart-type
            ecartType: {$stdDevPop: "$age"},
            
            // Écart-type échantillon
            ecartTypeEchantillon: {$stdDevSamp: "$age"}
        }
    }
])
\end{lstlisting}

\subsection{Group Multi-Niveaux}

\begin{lstlisting}
db.collection.aggregate([
    {
        $group: {
            _id: {
                ville: "$ville",
                age: "$age"
            },
            nombre: {$sum: 1}
        }
    },
    {
        $group: {
            _id: "$_id.ville",
            groupes: {$push: {
                age: "$_id.age",
                nombre: "$nombre"
            }}
        }
    }
])
\end{lstlisting}

\subsection{Pipeline Complexe - Exemple Complet}

\begin{lstlisting}
db.ventes.aggregate([
    // Étape 1: Filtrer
    {
        $match: {
            date: {$gte: ISODate("2024-01-01")},
            montant: {$gt: 0}
        }
    },
    
    // Étape 2: Déplier les produits
    {
        $unwind: "$produits"
    },
    
    // Étape 3: Calculer le total par produit
    {
        $project: {
            produitId: "$produits.id",
            quantite: "$produits.quantite",
            prixUnitaire: "$produits.prix",
            total: {
                $multiply: [
                    "$produits.quantite",
                    "$produits.prix"
                ]
            },
            date: 1
        }
    },
    
    // Étape 4: Regrouper par produit
    {
        $group: {
            _id: "$produitId",
            quantiteTotale: {$sum: "$quantite"},
            revenuTotal: {$sum: "$total"},
            nombreVentes: {$sum: 1},
            prixMoyen: {$avg: "$prixUnitaire"}
        }
    },
    
    // Étape 5: Trier par revenu
    {
        $sort: {revenuTotal: -1}
    },
    
    // Étape 6: Limiter aux 10 premiers
    {
        $limit: 10
    },
    
    // Étape 7: Formater la sortie
    {
        $project: {
            _id: 0,
            produitId: "$_id",
            quantiteTotale: 1,
            revenuTotal: {
                $round: ["$revenuTotal", 2]
            },
            nombreVentes: 1,
            prixMoyen: {
                $round: ["$prixMoyen", 2]
            }
        }
    }
])
\end{lstlisting}

\newpage

\section{Index et Performance}

\subsection{Création d'Index}

\begin{lstlisting}
// Index simple
db.collection.createIndex({nom: 1})

// Index composé
db.collection.createIndex({ville: 1, age: -1})

// Index unique
db.collection.createIndex({email: 1}, {unique: true})

// Index sur champ imbriqué
db.collection.createIndex({"adresse.ville": 1})

// Index sur tableau
db.collection.createIndex({tags: 1})
\end{lstlisting}

\subsection{Analyse de Performance}

\begin{lstlisting}
// Expliquer une requête
db.collection.find({age: 30}).explain("executionStats")

// Analyser le plan d'exécution
db.collection.find({age: 30}).explain("queryPlanner")
\end{lstlisting}

\newpage

\section{Exemples Pratiques}

\subsection{Exemple 1 : Statistiques par Ville}

\begin{lstlisting}
db.personnes.aggregate([
    {
        $group: {
            _id: "$ville",
            nombre: {$sum: 1},
            ageMoyen: {$avg: "$age"},
            ageMin: {$min: "$age"},
            ageMax: {$max: "$age"}
        }
    },
    {
        $sort: {nombre: -1}
    }
])
\end{lstlisting}

\subsection{Exemple 2 : Top 5 des Produits les Plus Vendus}

\begin{lstlisting}
db.ventes.aggregate([
    {
        $unwind: "$produits"
    },
    {
        $group: {
            _id: "$produits.nom",
            quantiteTotale: {$sum: "$produits.quantite"},
            revenuTotal: {
                $sum: {
                    $multiply: [
                        "$produits.quantite",
                        "$produits.prix"
                    ]
                }
            }
        }
    },
    {
        $sort: {quantiteTotale: -1}
    },
    {
        $limit: 5
    }
])
\end{lstlisting}

\subsection{Exemple 3 : Requête avec Jointure}

\begin{lstlisting}
db.commandes.aggregate([
    {
        $lookup: {
            from: "clients",
            localField: "clientId",
            foreignField: "_id",
            as: "client"
        }
    },
    {
        $unwind: "$client"
    },
    {
        $project: {
            numeroCommande: 1,
            date: 1,
            montant: 1,
            clientNom: "$client.nom",
            clientVille: "$client.ville"
        }
    }
])
\end{lstlisting}

\subsection{Exemple 4 : Calcul de Statistiques Mensuelles}

\begin{lstlisting}
db.ventes.aggregate([
    {
        $project: {
            annee: {$year: "$date"},
            mois: {$month: "$date"},
            montant: 1
        }
    },
    {
        $group: {
            _id: {
                annee: "$annee",
                mois: "$mois"
            },
            nombreVentes: {$sum: 1},
            montantTotal: {$sum: "$montant"},
            montantMoyen: {$avg: "$montant"}
        }
    },
    {
        $sort: {
            "_id.annee": 1,
            "_id.mois": 1
        }
    }
])
\end{lstlisting}

\newpage

\section{Résumé des Opérations Essentielles}

\subsection{CRUD}

\begin{table}[h]
\centering
\begin{tabular}{|c|c|}
\hline
\textbf{Opération} & \textbf{Commande} \\
\hline
Insert & \texttt{insertOne()}, \texttt{insertMany()} \\
\hline
Read & \texttt{find()}, \texttt{findOne()} \\
\hline
Update & \texttt{updateOne()}, \texttt{updateMany()} \\
\hline
Delete & \texttt{deleteOne()}, \texttt{deleteMany()} \\
\hline
\end{tabular}
\end{table}

\subsection{Opérateurs de Filtrage}

\begin{table}[h]
\centering
\begin{tabular}{|c|c|}
\hline
\textbf{Opérateur} & \textbf{Description} \\
\hline
\texttt{\$gt} & Plus grand que \\
\hline
\texttt{\$gte} & Plus grand ou égal \\
\hline
\texttt{\$lt} & Plus petit que \\
\hline
\texttt{\$lte} & Plus petit ou égal \\
\hline
\texttt{\$in} & Dans une liste \\
\hline
\texttt{\$nin} & Pas dans une liste \\
\hline
\texttt{\$and} & ET logique \\
\hline
\texttt{\$or} & OU logique \\
\hline
\texttt{\$regex} & Expression régulière \\
\hline
\end{tabular}
\end{table}

\subsection{Stages d'Agrégation}

\begin{table}[h]
\centering
\begin{tabular}{|c|c|}
\hline
\textbf{Stage} & \textbf{Description} \\
\hline
\texttt{\$match} & Filtrer les documents \\
\hline
\texttt{\$project} & Sélectionner/transformer les champs \\
\hline
\texttt{\$group} & Regrouper et agréger \\
\hline
\texttt{\$sort} & Trier les résultats \\
\hline
\texttt{\$limit} & Limiter le nombre de résultats \\
\hline
\texttt{\$skip} & Ignorer des résultats \\
\hline
\texttt{\$unwind} & Déplier un tableau \\
\hline
\texttt{\$lookup} & Effectuer une jointure \\
\hline
\texttt{\$addFields} & Ajouter des champs calculés \\
\hline
\texttt{\$count} & Compter les documents \\
\hline
\end{tabular}
\end{table}

\subsection{Accumulateurs de Group}

\begin{table}[h]
\centering
\begin{tabular}{|c|c|}
\hline
\textbf{Accumulateur} & \textbf{Description} \\
\hline
\texttt{\$sum} & Somme \\
\hline
\texttt{\$avg} & Moyenne \\
\hline
\texttt{\$min} & Minimum \\
\hline
\texttt{\$max} & Maximum \\
\hline
\texttt{\$first} & Premier élément \\
\hline
\texttt{\$last} & Dernier élément \\
\hline
\texttt{\$push} & Ajouter à un tableau \\
\hline
\texttt{\$addToSet} & Ajouter à un ensemble \\
\hline
\texttt{\$stdDevPop} & Écart-type population \\
\hline
\texttt{\$stdDevSamp} & Écart-type échantillon \\
\hline
\end{tabular}
\end{table}

\end{document}

