\documentclass[11pt,a4paper]{article}
\usepackage[utf8]{inputenc}
\usepackage[french]{babel}
\usepackage{amsmath}
\usepackage{amssymb}
\usepackage{amsthm}
\usepackage{mathtools}
\usepackage{geometry}
\usepackage{enumitem}
\usepackage{xcolor}
\usepackage{titlesec}
\usepackage{fancyhdr}
\usepackage{booktabs}
\usepackage{array}
\usepackage{longtable}

% Configuration de la page
\geometry{margin=2.5cm}
\pagestyle{fancy}
\fancyhf{}
\fancyhead[L]{\leftmark}
\fancyhead[R]{Chapitre 5 (Suite)}
\fancyfoot[C]{\thepage}

% Configuration des titres
\titleformat{\section}
{\Large\bfseries\color{blue!70!black}}
{}
{0em}
{}[\titlerule]

\titleformat{\subsection}
{\large\bfseries\color{blue!50!black}}
{}
{0em}
{}

\titleformat{\subsubsection}
{\normalsize\bfseries}
{}
{0em}
{}

% Commandes personnalisées
\newcommand{\R}{\mathbb{R}}
\newcommand{\Var}{\text{Var}}

% Métadonnées
\title{Résumé Complet - Chapitre 5 (Suite)\\
\large Tests d'Hypothèses}
\author{AmineGR03}
\date{\today}

\begin{document}

\maketitle

\tableofcontents
\newpage

\section{Introduction}

\subsection{Définition}

Un \textbf{test d'hypothèse} est une méthode statistique permettant de vérifier :
\begin{itemize}
    \item La valeur d'un paramètre (moyenne, variance, proportion, etc.)
    \item L'égalité des paramètres de deux distributions
    \item La forme d'une distribution
\end{itemize}

\subsection{Types d'hypothèses}

\begin{itemize}
    \item \textbf{Hypothèse paramétrique} : concerne la valeur d'un paramètre ou l'égalité de paramètres
    \item \textbf{Hypothèse non paramétrique} : concerne la forme d'une distribution
\end{itemize}

\newpage

\section{Hypothèses Nulle et Alternative}

\subsection{Définitions}

On cherche à vérifier la valeur d'un paramètre inconnu $\theta$ de la distribution d'une population $X$.

\textbf{Hypothèse nulle} ($H_0$) : 
\begin{align}
H_0 : \theta = \theta_0
\end{align}

\textbf{Hypothèse alternative} ($H_1$) : peut prendre l'une des formes suivantes :

\begin{enumerate}
    \item \textbf{Bilatérale} : $H_1 : \theta \neq \theta_0$
    \item \textbf{Unilatérale à gauche} : $H_1 : \theta < \theta_0$
    \item \textbf{Unilatérale à droite} : $H_1 : \theta > \theta_0$
\end{enumerate}

\subsection{Origine de l'hypothèse nulle}

L'hypothèse nulle peut provenir de :
\begin{itemize}
    \item Expériences antérieures (vérifier si les conditions ont changé)
    \item Une théorie ou un modèle (vérifier si le modèle est valide)
    \item Spécifications techniques (vérifier la conformité)
\end{itemize}

\newpage

\section{Erreurs de Type I et Type II}

\subsection{Définitions}

\begin{table}[h]
\centering
\begin{tabular}{|c|c|c|}
\hline
\textbf{Décision} & \textbf{Réalité : $H_0$ vraie} & \textbf{Réalité : $H_0$ fausse} \\
\hline
Accepter $H_0$ & $1-\alpha$ (correct) & Erreur de type II ($\beta$) \\
\hline
Rejeter $H_0$ & Erreur de type I ($\alpha$) & $1-\beta$ (puissance) \\
\hline
\end{tabular}
\end{table}

\textbf{Erreur de type I} :
\begin{align}
\alpha = P(\text{rejeter } H_0 | H_0 \text{ est vraie})
\end{align}

\textbf{Erreur de type II} :
\begin{align}
\beta = P(\text{accepter } H_0 | H_0 \text{ est fausse})
\end{align}

\subsection{Concepts importants}

\begin{itemize}
    \item \textbf{Seuil de signification} ($\alpha$) : probabilité de l'erreur de type I (fixé à l'avance, souvent 0.05 ou 0.01)
    \item \textbf{Puissance du test} : $1-\beta$ = probabilité de rejeter $H_0$ si $H_0$ est fausse
\end{itemize}

\subsection{Remarques}

\begin{itemize}
    \item On choisit généralement $\alpha$ petit (0.05, 0.01)
    \item Pour $\alpha$ et $n$ fixés, $\beta = \beta(\theta)$ est une fonction du paramètre $\theta$
    \item Le graphe de $\beta(\theta)$ est appelé \textbf{courbe caractéristique du test}
    \item Souvent $\alpha < \beta$
    \item Pour diminuer $\beta$, il faut augmenter $\alpha$ et/ou $n$
\end{itemize}

\newpage

\section{Région Critique}

\subsection{Principe}

Une \textbf{région critique} est une région où il est peu probable que la statistique prenne des valeurs lorsque l'hypothèse nulle est vraie.

\subsection{Décision}

\begin{itemize}
    \item \textbf{Rejeter $H_0$} si la valeur calculée de la statistique est dans la région critique $\rightarrow$ \textbf{conclusion forte}
    \item \textbf{Ne pas rejeter (accepter) $H_0$} si la valeur calculée est en dehors de la région critique $\rightarrow$ \textbf{conclusion faible}
\end{itemize}

\newpage

\section{Étapes d'Exécution d'un Test}

\begin{enumerate}
    \item \textbf{Formuler} $H_0$ et $H_1$
    \item \textbf{Choisir} $\alpha$ (seuil de signification)
    \item \textbf{Considérer} un échantillon de taille $n$
    \item \textbf{Exécuter} le test : calculer la statistique et vérifier si elle est dans la région critique
    \item \textbf{Conclure} en acceptant ou en rejetant $H_0$
    \item (Facultatif) Calculer $\beta$ et le risque de deuxième espèce
    \item (Facultatif) Si $\beta$ est trop élevé, indiquer un nouveau $\alpha$ et/ou un nouveau $n$ et recommencer
\end{enumerate}

\newpage

\section{Tests d'Hypothèses sur la Moyenne d'un Seul Échantillon}

\subsection{Test bilatéral : variance connue}

\subsubsection{Hypothèses}

\begin{align}
H_0 : \mu = \mu_0 \quad \text{vs} \quad H_1 : \mu \neq \mu_0
\end{align}

\subsubsection{Statistique de test}

\begin{align}
Z_0 = \frac{\bar{X} - \mu_0}{\sigma/\sqrt{n}}
\end{align}

Si $H_0$ est vraie, alors $Z_0 \sim N(0,1)$ (pour $n$ grand ou si $X \sim N(\mu, \sigma^2)$).

\subsubsection{Règle de décision}

\begin{itemize}
    \item \textbf{Rejeter $H_0$} si $|Z_0| > z_{\alpha/2}$
    \item \textbf{Accepter $H_0$} si $|Z_0| \leq z_{\alpha/2}$
\end{itemize}

où $z_{\alpha/2}$ est le quantile d'ordre $1-\alpha/2$ de la loi normale standard.

\textbf{Valeurs critiques usuelles} :
\begin{itemize}
    \item $\alpha = 0.05$ : $z_{0.025} = 1.96$
    \item $\alpha = 0.01$ : $z_{0.005} = 2.576$
\end{itemize}

\subsection{Test bilatéral : variance inconnue}

\subsubsection{Hypothèses}

\begin{align}
H_0 : \mu = \mu_0 \quad \text{vs} \quad H_1 : \mu \neq \mu_0
\end{align}

\subsubsection{Statistique de test}

\begin{align}
T_0 = \frac{\bar{X} - \mu_0}{S/\sqrt{n}}
\end{align}

où $S$ est l'écart-type empirique :

\begin{align}
S = \sqrt{\frac{1}{n-1}\sum_{i=1}^{n}(x_i - \bar{x})^2}
\end{align}

Si $H_0$ est vraie et $X \sim N(\mu, \sigma^2)$, alors $T_0 \sim t_{n-1}$ (loi de Student à $n-1$ degrés de liberté).

\subsubsection{Règle de décision}

\begin{itemize}
    \item \textbf{Rejeter $H_0$} si $|T_0| > t_{\alpha/2; n-1}$
    \item \textbf{Accepter $H_0$} si $|T_0| \leq t_{\alpha/2; n-1}$
\end{itemize}

où $t_{\alpha/2; n-1}$ est le quantile d'ordre $1-\alpha/2$ de la loi de Student à $n-1$ degrés de liberté.

\subsection{Tests unilatéraux}

\subsubsection{Test unilatéral à gauche}

\textbf{Hypothèses} :
\begin{align}
H_0 : \mu = \mu_0 \text{ (ou } \mu \geq \mu_0\text{)}, \quad H_1 : \mu < \mu_0
\end{align}

\textbf{Règle de rejet} :
\begin{itemize}
    \item Si $\sigma$ connu : rejeter $H_0$ si $Z_0 < -z_{\alpha}$
    \item Si $\sigma$ inconnu : rejeter $H_0$ si $T_0 < -t_{\alpha; n-1}$
\end{itemize}

\subsubsection{Test unilatéral à droite}

\textbf{Hypothèses} :
\begin{align}
H_0 : \mu = \mu_0 \text{ (ou } \mu \leq \mu_0\text{)}, \quad H_1 : \mu > \mu_0
\end{align}

\textbf{Règle de rejet} :
\begin{itemize}
    \item Si $\sigma$ connu : rejeter $H_0$ si $Z_0 > z_{\alpha}$
    \item Si $\sigma$ inconnu : rejeter $H_0$ si $T_0 > t_{\alpha; n-1}$
\end{itemize}

\subsubsection{Tableau récapitulatif}

\begin{table}[h]
\centering
\begin{tabular}{|c|c|c|c|}
\hline
\textbf{Test} & \textbf{$H_1$} & \textbf{$\sigma$ connu} & \textbf{$\sigma$ inconnu} \\
\hline
Unilatéral gauche & $\mu < \mu_0$ & $Z_0 < -z_{\alpha}$ & $T_0 < -t_{\alpha; n-1}$ \\
\hline
Unilatéral droite & $\mu > \mu_0$ & $Z_0 > z_{\alpha}$ & $T_0 > t_{\alpha; n-1}$ \\
\hline
Bilatéral & $\mu \neq \mu_0$ & $|Z_0| > z_{\alpha/2}$ & $|T_0| > t_{\alpha/2; n-1}$ \\
\hline
\end{tabular}
\end{table}

\newpage

\section{Niveau Critique Observé (P-value)}

\subsection{Définition}

Le \textbf{niveau critique observé} (ou \textbf{P-value}, notée $PV$) est la valeur minimale de $\alpha$ telle que $H_0$ est toujours rejetée.

\subsection{Avantages}

\begin{itemize}
    \item Une fois la P-value connue, on peut déterminer la décision pour n'importe quel seuil $\alpha$
    \item Si $\alpha > PV$, alors $H_0$ est rejetée
    \item Les logiciels donnent généralement la P-value
\end{itemize}

\subsection{Calcul de la P-value}

Soit $Z_0$ (ou $T_0$) la statistique employée et $z_0$ (ou $t_0$) sa valeur calculée.

\subsubsection{Si $\sigma^2$ est connue (test sur la moyenne)}

\begin{itemize}
    \item \textbf{Unilatéral gauche} ($H_1 : \mu < \mu_0$) : $PV = \Phi(z_0)$
    \item \textbf{Unilatéral droite} ($H_1 : \mu > \mu_0$) : $PV = 1 - \Phi(z_0)$
    \item \textbf{Bilatéral} ($H_1 : \mu \neq \mu_0$) : $PV = 2(1 - \Phi(|z_0|))$
\end{itemize}

où $\Phi$ est la fonction de répartition de la loi normale standard.

\subsubsection{Si $\sigma^2$ est inconnue (test sur la moyenne)}

\begin{itemize}
    \item \textbf{Unilatéral} : $PV = P(T > |t_0|)$ avec $T \sim t_{n-1}$
    \item \textbf{Bilatéral} : $PV = 2P(T > |t_0|)$ avec $T \sim t_{n-1}$
\end{itemize}

\subsection{Interprétation}

\begin{itemize}
    \item Si $PV < \alpha$ : rejeter $H_0$
    \item Si $PV \geq \alpha$ : ne pas rejeter $H_0$
\end{itemize}

\newpage

\section{Test d'Ajustement du Khi-deux ($\chi^2$)}

\subsection{Objectif}

Vérifier si les données $x_1, \ldots, x_n$ proviennent d'une population distribuée selon une loi particulière $F(x, \theta)$.

\subsection{Hypothèses}

\begin{align}
H_0 : X \sim F(x, \theta) \quad \text{vs} \quad H_1 : X \not\sim F(x, \theta)
\end{align}

\subsection{Méthode}

\subsubsection{Étape 1 : Regroupement des observations}

On regroupe les observations en $k$ classes (valeurs ou intervalles) :

\begin{table}[h]
\centering
\begin{tabular}{|c|c|c|c|c|c|}
\hline
\textbf{Valeurs ($V_i$)} & $V_1$ & $V_2$ & $\ldots$ & $V_i$ & $\ldots$ \\
\hline
\textbf{Effectifs observés ($O_i$)} & $O_1$ & $O_2$ & $\ldots$ & $O_i$ & $\ldots$ \\
\hline
\textbf{Effectifs attendus ($E_i$)} & $E_1$ & $E_2$ & $\ldots$ & $E_i$ & $\ldots$ \\
\hline
\textbf{Total} & & & & & $n$ \\
\hline
\end{tabular}
\end{table}

\subsubsection{Étape 2 : Calcul des effectifs attendus}

\begin{align}
E_i = n \times p_i^{(0)}
\end{align}

où :
\begin{align}
p_i^{(0)} = P(X \in V_i | H_0 \text{ est vraie}), \quad i = 1, 2, \ldots, k
\end{align}

et :
\begin{align}
\sum_{i=1}^{k} p_i^{(0)} = 1
\end{align}

\textbf{Remarque} : Si certains $E_i$ sont petits ($< 5$), regrouper des classes.

\subsubsection{Étape 3 : Calcul de la statistique du test}

\begin{align}
\chi_0^2 = \sum_{i=1}^{k} \frac{(O_i - E_i)^2}{E_i}
\end{align}

La statistique $\chi_0^2$ représente une "distance" globale entre les effectifs observés et attendus.

\subsubsection{Étape 4 : Distribution sous $H_0$}

Si $H_0$ est vraie, alors $\chi_0^2 \sim \chi_{\nu}^2$ (loi du khi-deux) avec :

\begin{align}
\nu = k - p - 1
\end{align}

où :
\begin{itemize}
    \item $k$ : nombre de classes retenues
    \item $p$ : nombre de paramètres estimés
\end{itemize}

\subsubsection{Étape 5 : Règle de décision}

Pour un niveau critique $\alpha$ donné :

\begin{itemize}
    \item \textbf{Rejeter $H_0$} si $\chi_0^2 > \chi_{\alpha; \nu}^2$
    \item \textbf{Accepter $H_0$} si $\chi_0^2 \leq \chi_{\alpha; \nu}^2$
\end{itemize}

où $\chi_{\alpha; \nu}^2$ est le quantile d'ordre $1-\alpha$ de la loi $\chi_{\nu}^2$.

\newpage

\section{Résumé des Formules Clés}

\subsection{Tests sur la moyenne (variance connue)}

\begin{align}
Z_0 = \frac{\bar{X} - \mu_0}{\sigma/\sqrt{n}} \sim N(0,1)
\end{align}

\subsection{Tests sur la moyenne (variance inconnue)}

\begin{align}
T_0 = \frac{\bar{X} - \mu_0}{S/\sqrt{n}} \sim t_{n-1}
\end{align}

où :
\begin{align}
S = \sqrt{\frac{1}{n-1}\sum_{i=1}^{n}(x_i - \bar{x})^2}
\end{align}

\subsection{Test d'ajustement du khi-deux}

\begin{align}
\chi_0^2 = \sum_{i=1}^{k} \frac{(O_i - E_i)^2}{E_i} \sim \chi_{k-p-1}^2
\end{align}

\subsection{P-value (cas bilatéral, variance connue)}

\begin{align}
PV = 2(1 - \Phi(|z_0|))
\end{align}

\newpage

\section{Exercice d'Application}

\subsection{Énoncé}

Une entreprise fabrique des boulons. Le cahier des charges spécifie que le diamètre moyen des boulons doit être de 10 mm avec un écart-type de 0.5 mm.

Un contrôle qualité est effectué sur un échantillon de 25 boulons. Les résultats sont les suivants :

\textbf{Diamètres mesurés (en mm)} :
9.8, 10.1, 9.9, 10.2, 9.7, 10.0, 10.3, 9.8, 10.1, 9.9, 10.0, 10.2, 9.8, 10.1, 10.0, 9.9, 10.2, 10.1, 9.8, 10.0, 10.1, 9.9, 10.2, 10.0, 9.8

\textbf{Questions} :

\begin{enumerate}
    \item \textbf{Test bilatéral} : Au seuil de signification $\alpha = 0.05$, peut-on conclure que le diamètre moyen est conforme aux spécifications (10 mm) ?
    \item \textbf{Test unilatéral} : Au seuil $\alpha = 0.05$, peut-on conclure que le diamètre moyen est supérieur à 10 mm ?
    \item \textbf{P-value} : Calculer la P-value pour le test bilatéral.
    \item \textbf{Test d'ajustement} : On souhaite vérifier si les diamètres suivent une loi normale $N(10, 0.25)$. Les données ont été regroupées en 4 classes :
    
    \begin{table}[h]
    \centering
    \begin{tabular}{|c|c|c|c|c|}
    \hline
    \textbf{Intervalle} & $[9.5, 9.75[$ & $[9.75, 10[$ & $[10, 10.25[$ & $[10.25, 10.5[$ \\
    \hline
    \textbf{Effectifs observés} & 3 & 8 & 10 & 4 \\
    \hline
    \end{tabular}
    \end{table}
    
    Effectuer le test d'ajustement du khi-deux au seuil $\alpha = 0.05$.
\end{enumerate}

\subsection{Solution guidée}

\subsubsection{1. Test bilatéral (variance connue)}

\textbf{Étape 1} : Formuler les hypothèses

\begin{align}
H_0 : \mu = 10 \quad \text{vs} \quad H_1 : \mu \neq 10
\end{align}

\textbf{Étape 2} : Calculer la moyenne de l'échantillon

\begin{align}
\bar{x} = \frac{1}{25}\sum_{i=1}^{25} x_i = \frac{249.8}{25} = 9.992 \text{ mm}
\end{align}

\textbf{Étape 3} : Calculer la statistique de test

\begin{align}
Z_0 = \frac{\bar{X} - \mu_0}{\sigma/\sqrt{n}} = \frac{9.992 - 10}{0.5/\sqrt{25}} = \frac{-0.008}{0.1} = -0.08
\end{align}

\textbf{Étape 4} : Règle de décision

Pour $\alpha = 0.05$, on a $z_{0.025} = 1.96$.

\begin{align}
|Z_0| = 0.08 < 1.96
\end{align}

\textbf{Conclusion} : On \textbf{accepte $H_0$} au seuil 5\%. Le diamètre moyen est conforme aux spécifications.

\subsubsection{2. Test unilatéral à droite}

\textbf{Étape 1} : Formuler les hypothèses

\begin{align}
H_0 : \mu \leq 10 \quad \text{vs} \quad H_1 : \mu > 10
\end{align}

\textbf{Étape 2} : Statistique de test (même calcul)

\begin{align}
Z_0 = -0.08
\end{align}

\textbf{Étape 3} : Règle de décision

Pour $\alpha = 0.05$, on a $z_{0.05} = 1.645$.

\begin{align}
Z_0 = -0.08 < 1.645
\end{align}

\textbf{Conclusion} : On \textbf{accepte $H_0$} au seuil 5\%. On ne peut pas conclure que le diamètre moyen est supérieur à 10 mm.

\subsubsection{3. Calcul de la P-value (test bilatéral)}

\begin{align}
PV = 2(1 - \Phi(|z_0|)) = 2(1 - \Phi(0.08)) = 2(1 - 0.5319) = 2 \times 0.4681 = 0.9362
\end{align}

\textbf{Interprétation} : $PV = 0.9362 > 0.05$, donc on accepte $H_0$ pour tout seuil $\alpha < 0.9362$.

\subsubsection{4. Test d'ajustement du khi-deux}

\textbf{Étape 1} : Hypothèses

\begin{align}
H_0 : X \sim N(10, 0.25) \quad \text{vs} \quad H_1 : X \not\sim N(10, 0.25)
\end{align}

\textbf{Étape 2} : Calculer les effectifs attendus

Pour chaque intervalle, on calcule $p_i^{(0)} = P(X \in \text{intervalle} | H_0)$ avec $X \sim N(10, 0.5^2)$.

\begin{itemize}
    \item \textbf{Intervalle $[9.5, 9.75[$} :
    \begin{align}
    p_1 = P(9.5 \leq X < 9.75) = \Phi\left(\frac{9.75-10}{0.5}\right) - \Phi\left(\frac{9.5-10}{0.5}\right)
    \end{align}
    \begin{align}
    = \Phi(-0.5) - \Phi(-1) = (1 - 0.6915) - (1 - 0.8413) = 0.1498
    \end{align}
    \begin{align}
    E_1 = 25 \times 0.1498 = 3.745
    \end{align}
    
    \item \textbf{Intervalle $[9.75, 10[$} :
    \begin{align}
    p_2 = P(9.75 \leq X < 10) = \Phi(0) - \Phi(-0.5) = 0.5 - 0.3085 = 0.1915
    \end{align}
    \begin{align}
    E_2 = 25 \times 0.1915 = 4.788
    \end{align}
    
    \item \textbf{Intervalle $[10, 10.25[$} :
    \begin{align}
    p_3 = P(10 \leq X < 10.25) = \Phi(0.5) - \Phi(0) = 0.6915 - 0.5 = 0.1915
    \end{align}
    \begin{align}
    E_3 = 25 \times 0.1915 = 4.788
    \end{align}
    
    \item \textbf{Intervalle $[10.25, 10.5[$} :
    \begin{align}
    p_4 = P(10.25 \leq X < 10.5) = \Phi(1) - \Phi(0.5) = 0.8413 - 0.6915 = 0.1498
    \end{align}
    \begin{align}
    E_4 = 25 \times 0.1498 = 3.745
    \end{align}
\end{itemize}

\textbf{Recalcul correct} : On inclut $P(X < 9.5)$ et $P(X \geq 10.5)$ dans les classes extrêmes.

\begin{align}
P(X < 9.5) = \Phi(-1) = 0.1587, \quad P(X \geq 10.5) = 1 - \Phi(1) = 0.1587
\end{align}

En regroupant :
\begin{itemize}
    \item Classe 1 : $X < 9.75$ $\rightarrow$ $E_1 = 25 \times (0.1587 + 0.1498) = 7.7125$
    \item Classe 2 : $9.75 \leq X < 10$ $\rightarrow$ $E_2 = 4.788$
    \item Classe 3 : $10 \leq X < 10.25$ $\rightarrow$ $E_3 = 4.788$
    \item Classe 4 : $X \geq 10.25$ $\rightarrow$ $E_4 = 25 \times (0.1498 + 0.1587) = 7.7125$
\end{itemize}

\textbf{Étape 3} : Calculer $\chi_0^2$

\begin{align}
\chi_0^2 = \sum_{i=1}^{4} \frac{(O_i - E_i)^2}{E_i}
\end{align}

\begin{align}
= \frac{(3 - 7.7125)^2}{7.7125} + \frac{(8 - 4.788)^2}{4.788} + \frac{(10 - 4.788)^2}{4.788} + \frac{(4 - 7.7125)^2}{7.7125}
\end{align}

\begin{align}
= \frac{22.19}{7.7125} + \frac{10.32}{4.788} + \frac{27.17}{4.788} + \frac{13.79}{7.7125}
\end{align}

\begin{align}
= 2.88 + 2.15 + 5.68 + 1.79 = 12.50
\end{align}

\textbf{Étape 4} : Degrés de liberté

\begin{align}
\nu = k - p - 1 = 4 - 0 - 1 = 3
\end{align}

(Aucun paramètre estimé car $\mu = 10$ et $\sigma = 0.5$ sont connus)

\textbf{Étape 5} : Règle de décision

Pour $\alpha = 0.05$ et $\nu = 3$, on a $\chi_{0.05; 3}^2 \approx 7.815$.

\begin{align}
\chi_0^2 = 12.50 > 7.815
\end{align}

\textbf{Conclusion} : On \textbf{rejette $H_0$} au seuil 5\%. Les diamètres ne suivent pas une loi normale $N(10, 0.25)$.

\newpage

\section{Valeurs Critiques Usuelles}

\subsection{Loi normale standard}

\begin{table}[h]
\centering
\begin{tabular}{|c|c|c|}
\hline
$\alpha$ & $z_{\alpha}$ & $z_{\alpha/2}$ \\
\hline
$0.10$ & $1.282$ & $1.645$ \\
\hline
$0.05$ & $1.645$ & $1.960$ \\
\hline
$0.01$ & $2.326$ & $2.576$ \\
\hline
\end{tabular}
\end{table}

\subsection{Loi de Student (exemples)}

\begin{table}[h]
\centering
\begin{tabular}{|c|c|c|}
\hline
\textbf{ddl} & $t_{0.05}$ & $t_{0.025}$ \\
\hline
$10$ & $1.812$ & $2.228$ \\
\hline
$20$ & $1.725$ & $2.086$ \\
\hline
$30$ & $1.697$ & $2.042$ \\
\hline
$\infty$ & $1.645$ & $1.960$ \\
\hline
\end{tabular}
\end{table}

\subsection{Loi du khi-deux (exemples)}

\begin{table}[h]
\centering
\begin{tabular}{|c|c|c|}
\hline
\textbf{ddl} & $\chi_{0.05}^2$ & $\chi_{0.01}^2$ \\
\hline
$1$ & $3.841$ & $6.635$ \\
\hline
$2$ & $5.991$ & $9.210$ \\
\hline
$3$ & $7.815$ & $11.345$ \\
\hline
$5$ & $11.070$ & $15.086$ \\
\hline
$10$ & $18.307$ & $23.209$ \\
\hline
\end{tabular}
\end{table}

\end{document}

