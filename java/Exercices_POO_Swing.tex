\documentclass[11pt,a4paper]{article}
\usepackage[utf8]{inputenc}
\usepackage[french]{babel}
\usepackage{amsmath}
\usepackage{amssymb}
\usepackage{geometry}
\usepackage{enumitem}
\usepackage{xcolor}
\usepackage{titlesec}
\usepackage{fancyhdr}
\usepackage{booktabs}
\usepackage{array}
\usepackage{listings}
\usepackage{tcolorbox}

% Configuration de la page
\geometry{margin=2.5cm}
\pagestyle{fancy}
\fancyhf{}
\fancyhead[L]{\leftmark}
\fancyhead[R]{Exercices POO et Swing}
\fancyfoot[C]{\thepage}

% Configuration des titres
\titleformat{\section}
{\Large\bfseries\color{blue!70!black}}
{}
{0em}
{}[\titlerule]

\titleformat{\subsection}
{\large\bfseries\color{blue!50!black}}
{}
{0em}
{}

% Configuration pour le code Java
\lstset{
    language=Java,
    basicstyle=\ttfamily\small,
    keywordstyle=\color{blue}\bfseries,
    commentstyle=\color{gray}\itshape,
    stringstyle=\color{red},
    numbers=left,
    numberstyle=\tiny\color{gray},
    stepnumber=1,
    numbersep=5pt,
    backgroundcolor=\color{gray!10},
    frame=single,
    breaklines=true,
    showstringspaces=false,
    tabsize=4
}

% Configuration pour les zones de réponse
\tcbuselibrary{breakable}
\newtcolorbox{reponsebox}{
    colback=gray!5,
    colframe=gray!50,
    breakable,
    title=Réponse
}

% Métadonnées
\title{Exercices - POO et Swing\\
\large Programmation Orientée Objet et Interfaces Graphiques}
\author{AmineGR03}
\date{\today}

\begin{document}

\maketitle

\tableofcontents
\newpage

\section{Exercice 1 : Questions Directes en POO}

\subsection{Question 1}

\textbf{Quelle est la différence entre une classe abstraite et une interface en Java ? Donnez un exemple concret pour illustrer cette différence.}

\begin{reponsebox}
\vspace{3cm}
\end{reponsebox}

\subsection{Question 2}

\textbf{Expliquez le concept de polymorphisme en POO. Donnez un exemple de code Java qui illustre le polymorphisme avec au moins deux classes.}

\begin{reponsebox}
\vspace{4cm}
\end{reponsebox}

\subsection{Question 3}

\textbf{Qu'est-ce qu'une classe générique (generic class) en Java ? Pourquoi les utiliser ? Donnez un exemple de classe générique avec deux paramètres de type.}

\begin{reponsebox}
\vspace{4cm}
\end{reponsebox}

\subsection{Question 4}

\textbf{Expliquez la différence entre \texttt{ArrayList} et \texttt{LinkedList} en Java. Dans quels cas utiliseriez-vous chacune de ces collections ?}

\begin{reponsebox}
\vspace{4cm}
\end{reponsebox}

\subsection{Question 5}

\textbf{Qu'est-ce qu'un stream en Java 8+ ? Donnez un exemple d'utilisation d'un stream pour filtrer et transformer une liste d'entiers.}

\begin{reponsebox}
\vspace{4cm}
\end{reponsebox}

\newpage

\section{Exercice 2 : Identifier les Erreurs et Proposer des Solutions}

\subsection{Erreur 1 : Héritage}

\textbf{Identifiez l'erreur dans le code suivant et proposez une solution :}

\begin{lstlisting}
class Animal {
    private String nom;
    
    public Animal(String nom) {
        this.nom = nom;
    }
    
    public void manger() {
        System.out.println(nom + " mange");
    }
}

class Chien extends Animal {
    public Chien(String nom) {
        // Erreur ici
    }
    
    public void aboyer() {
        System.out.println("Wouf wouf !");
    }
}

public class Test {
    public static void main(String[] args) {
        Chien chien = new Chien("Rex");
        chien.manger();
    }
}
\end{lstlisting}

\begin{reponsebox}
\vspace{5cm}
\end{reponsebox}

\subsection{Erreur 2 : Polymorphisme}

\textbf{Identifiez l'erreur dans le code suivant et proposez une solution :}

\begin{lstlisting}
class Forme {
    public void dessiner() {
        System.out.println("Dessiner une forme");
    }
}

class Cercle extends Forme {
    public void dessiner() {
        System.out.println("Dessiner un cercle");
    }
    
    public void calculerAire() {
        System.out.println("Aire du cercle");
    }
}

public class Test {
    public static void main(String[] args) {
        Forme forme = new Cercle();
        forme.dessiner();
        forme.calculerAire(); // Erreur ici
    }
}
\end{lstlisting}

\begin{reponsebox}
\vspace{5cm}
\end{reponsebox}

\subsection{Erreur 3 : Classes Génériques}

\textbf{Identifiez l'erreur dans le code suivant et proposez une solution :}

\begin{lstlisting}
class Boite<T> {
    private T contenu;
    
    public void setContenu(T contenu) {
        this.contenu = contenu;
    }
    
    public T getContenu() {
        return contenu;
    }
}

public class Test {
    public static void main(String[] args) {
        Boite<String> boiteString = new Boite<>();
        boiteString.setContenu("Hello");
        
        Boite<Integer> boiteInt = boiteString; // Erreur ici
        boiteInt.setContenu(42);
        
        System.out.println(boiteString.getContenu());
    }
}
\end{lstlisting}

\begin{reponsebox}
\vspace{5cm}
\end{reponsebox}

\subsection{Erreur 4 : Collections}

\textbf{Identifiez l'erreur dans le code suivant et proposez une solution :}

\begin{lstlisting}
import java.util.*;

public class Test {
    public static void main(String[] args) {
        List<String> liste = new ArrayList<>();
        liste.add("A");
        liste.add("B");
        liste.add("C");
        
        for (String element : liste) {
            if (element.equals("B")) {
                liste.remove(element); // Erreur ici
            }
        }
        
        System.out.println(liste);
    }
}
\end{lstlisting}

\begin{reponsebox}
\vspace{5cm}
\end{reponsebox}

\subsection{Erreur 5 : Stream}

\textbf{Identifiez l'erreur dans le code suivant et proposez une solution :}

\begin{lstlisting}
import java.util.*;
import java.util.stream.*;

public class Test {
    public static void main(String[] args) {
        List<Integer> nombres = Arrays.asList(1, 2, 3, 4, 5, null, 7);
        
        int somme = nombres.stream()
            .filter(n -> n > 0)
            .mapToInt(n -> n * 2) // Erreur ici
            .sum();
        
        System.out.println("Somme : " + somme);
    }
}
\end{lstlisting}

\begin{reponsebox}
\vspace{5cm}
\end{reponsebox}

\newpage

\section{Exercice 3 : Code à Compléter}

\textbf{Complétez le code suivant en remplissant les 10 champs manquants (marqués par \texttt{// TODO}). Comprenez la logique du programme avant de compléter.}

\begin{lstlisting}
import java.util.*;
import java.util.stream.*;

// TODO 1: Déclarer une interface fonctionnelle pour calculer
@FunctionalInterface
interface Calculateur {
    // TODO 2: Déclarer la méthode abstraite
    _____________________________
}

// TODO 3: Déclarer une classe générique
class _____________________________<T extends Number> {
    private List<T> elements;
    
    public _____________________________(List<T> elements) {
        this.elements = elements;
    }
    
    public double calculer(Calculateur calc) {
        return elements.stream()
            // TODO 4: Convertir en double
            ._____________________________
            // TODO 5: Appliquer le calcul
            ._____________________________
            // TODO 6: Obtenir la somme
            ._____________________________;
    }
    
    public List<T> filtrer(Predicate<T> condition) {
        return elements.stream()
            // TODO 7: Filtrer selon la condition
            ._____________________________
            // TODO 8: Collecter dans une liste
            ._____________________________;
    }
}

public class ExerciceComplet {
    public static void main(String[] args) {
        List<Integer> nombres = Arrays.asList(1, 2, 3, 4, 5, 6, 7, 8, 9, 10);
        
        // TODO 9: Créer une instance de la classe générique
        _____________________________ calculatrice = new _____________________________<>(nombres);
        
        // Calculer la somme des carrés
        double resultat = calculatrice.calculer(
            // TODO 10: Implémenter l'interface fonctionnelle avec lambda
            _____________________________
        );
        
        System.out.println("Somme des carrés : " + resultat);
        
        // Filtrer les nombres pairs
        List<Integer> pairs = calculatrice.filtrer(n -> n % 2 == 0);
        System.out.println("Nombres pairs : " + pairs);
    }
}
\end{lstlisting}

\textbf{Instructions :}
\begin{itemize}
    \item Le programme doit calculer la somme des carrés des nombres
    \item Il doit aussi filtrer les nombres pairs
    \item Utilisez les concepts de génériques, streams, et interfaces fonctionnelles
    \item Le résultat attendu : "Somme des carrés : 385.0" et "Nombres pairs : [2, 4, 6, 8, 10]"
\end{itemize}

\newpage

\section{Exercice 4 : Interface Swing - Identifier les Composants et Événements}

\subsection{Code à Analyser}

\textbf{Analysez le code suivant et répondez aux questions :}

\begin{lstlisting}
import javax.swing.*;
import java.awt.*;
import java.awt.event.ActionEvent;
import java.awt.event.ActionListener;

public class InterfaceSwing extends JFrame {
    private JTextField champNom;
    private JTextField champAge;
    private JLabel labelResultat;
    private JButton boutonValider;
    private JButton boutonEffacer;
    private JCheckBox checkBoxAccepte;
    private JComboBox<String> comboBoxVille;
    private JPanel panelPrincipal;
    private JPanel panelBoutons;
    
    public InterfaceSwing() {
        setTitle("Formulaire d'Inscription");
        setSize(500, 400);
        setLocationRelativeTo(null);
        setDefaultCloseOperation(JFrame.EXIT_ON_CLOSE);
        
        // Configuration du layout principal
        setLayout(new BorderLayout());
        
        // Panel principal
        panelPrincipal = new JPanel();
        panelPrincipal.setLayout(new GridLayout(4, 2, 10, 10));
        panelPrincipal.setBorder(BorderFactory.createTitledBorder("Informations"));
        
        // Composants du formulaire
        panelPrincipal.add(new JLabel("Nom :"));
        champNom = new JTextField(20);
        panelPrincipal.add(champNom);
        
        panelPrincipal.add(new JLabel("Âge :"));
        champAge = new JTextField(20);
        panelPrincipal.add(champAge);
        
        panelPrincipal.add(new JLabel("Ville :"));
        String[] villes = {"Paris", "Lyon", "Marseille", "Toulouse"};
        comboBoxVille = new JComboBox<>(villes);
        panelPrincipal.add(comboBoxVille);
        
        panelPrincipal.add(new JLabel("Accepter les conditions :"));
        checkBoxAccepte = new JCheckBox("J'accepte");
        panelPrincipal.add(checkBoxAccepte);
        
        // Panel des boutons
        panelBoutons = new JPanel();
        panelBoutons.setLayout(new FlowLayout());
        
        boutonValider = new JButton("Valider");
        boutonValider.addActionListener(new ActionListener() {
            @Override
            public void actionPerformed(ActionEvent e) {
                String nom = champNom.getText();
                String age = champAge.getText();
                String ville = (String) comboBoxVille.getSelectedItem();
                boolean accepte = checkBoxAccepte.isSelected();
                
                if (nom.isEmpty() || age.isEmpty()) {
                    labelResultat.setText("Veuillez remplir tous les champs !");
                    labelResultat.setForeground(Color.RED);
                } else if (!accepte) {
                    labelResultat.setText("Vous devez accepter les conditions !");
                    labelResultat.setForeground(Color.RED);
                } else {
                    labelResultat.setText("Inscription réussie : " + nom + 
                                         ", " + age + " ans, " + ville);
                    labelResultat.setForeground(Color.GREEN);
                }
            }
        });
        
        boutonEffacer = new JButton("Effacer");
        boutonEffacer.addActionListener(new ActionListener() {
            @Override
            public void actionPerformed(ActionEvent e) {
                champNom.setText("");
                champAge.setText("");
                comboBoxVille.setSelectedIndex(0);
                checkBoxAccepte.setSelected(false);
                labelResultat.setText("");
            }
        });
        
        panelBoutons.add(boutonValider);
        panelBoutons.add(boutonEffacer);
        
        // Label de résultat
        labelResultat = new JLabel(" ");
        labelResultat.setHorizontalAlignment(SwingConstants.CENTER);
        labelResultat.setFont(new Font("Arial", Font.BOLD, 12));
        
        // Ajout des composants à la fenêtre
        add(panelPrincipal, BorderLayout.CENTER);
        add(panelBoutons, BorderLayout.SOUTH);
        add(labelResultat, BorderLayout.NORTH);
        
        setVisible(true);
    }
    
    public static void main(String[] args) {
        SwingUtilities.invokeLater(() -> {
            new InterfaceSwing();
        });
    }
}
\end{lstlisting}

\subsection{Questions}

\subsubsection{Question 1 : Identification des Composants}

\textbf{Listez tous les composants Swing utilisés dans ce code et indiquez leur rôle :}

\begin{reponsebox}
\vspace{6cm}
\end{reponsebox}

\subsubsection{Question 2 : Layout Managers}

\textbf{Quels layout managers sont utilisés ? Dans quels conteneurs ? Expliquez leur rôle :}

\begin{reponsebox}
\vspace{5cm}
\end{reponsebox}

\subsubsection{Question 3 : Événements sur le Bouton "Valider"}

\textbf{Décrivez en détail ce qui se passe quand on clique sur le bouton "Valider" :}
\begin{itemize}
    \item Quel type d'événement est déclenché ?
    \item Quelle méthode est appelée ?
    \item Quelles actions sont effectuées ?
    \item Quelles sont les conditions de validation ?
\end{itemize}

\begin{reponsebox}
\vspace{8cm}
\end{reponsebox}

\subsubsection{Question 4 : Événements sur le Bouton "Effacer"}

\textbf{Décrivez en détail ce qui se passe quand on clique sur le bouton "Effacer" :}
\begin{itemize}
    \item Quel type d'événement est déclenché ?
    \item Quelle méthode est appelée ?
    \item Quelles actions sont effectuées ?
\end{itemize}

\begin{reponsebox}
\vspace{6cm}
\end{reponsebox}

\subsubsection{Question 5 : Structure de l'Interface}

\textbf{Dessinez un schéma (ou décrivez) la structure hiérarchique de l'interface :}
\begin{itemize}
    \item Quelle est la fenêtre principale ?
    \item Quels sont les conteneurs intermédiaires ?
    \item Comment sont organisés les composants ?
\end{itemize}

\begin{reponsebox}
\vspace{6cm}
\end{reponsebox}

\subsubsection{Question 6 : Améliorations Possibles}

\textbf{Proposez 3 améliorations possibles pour cette interface (validation, design, fonctionnalités) :}

\begin{reponsebox}
\vspace{5cm}
\end{reponsebox}

\end{document}






